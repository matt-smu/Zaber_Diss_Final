
The motivation of this thesis is to make modern information systems more secure, and the driving force behind that goal is automation. Many of the problems addressed in this work stem from the disparate ecosystem of tools, APIs, methodologies, libraries, and frameworks that exist in relative isolation to one another. Consider Security Information and Event Management (SIEM) systems as an example, which provide correlation of host/network event logs, IDS/IPS alerts, threat/vulnerability feeds, etc, and present a unified view of the system’s security posture automatically to the SOC. Before the advent of managed SIEMs, sys admins typically filled the role of security engineers, and relied on hand rolled collections of shell/perl scripts to manage systems, parse logs, collect or push events, format reports, and issue alarms. To be effective required tribal knowledge along with proficiency in programming, network plumbing, and systems management, so changes to the environment or workforce made it extremely difficult(expensive) to deliver continuous monitoring capabilities to operators at any scale. 

We are in a similar state today with network design and enterprise planning. Infrastructure-as-Code, SDN, virtualization and containerization are all critical components in modern deployments, but the glue that ties them together is largely ad-hoc, and risk evaluation is still a manual task. In order to understand the security posture before a system is rolled out and SIEMs are in place, we are creating a tool to facilitate the automated analysis, collection, correlation, and dissemination of the security metrics mentioned above. The necessity of such a tool is critical to evaluating the efficacy of the metrics reviewed above, and provides the foundation for ongoing research in machine learning models for secure systems planning, design, and evolution.


If we measure an aspect of security before and after a change takes place, then we can quantify the impact that change had on security. If we test an aspect of cyber security at regular intervals, then we can determine the average rate of change for that security metric over the given time period. In order to sample security measurements at regular (approaching continuous) intervals, we assert that the test apparatus must be fully automated. Using the 5 basic CyBOK categories as a guide, we can briefly identify some types and familiar sources of available security metrics and describe the suitability to automation for each.

% \begin{theorem}\label{theo:intro:secmet_diff}
% If we measure an aspect of security before and after a change takes place, then we can quantify the impact that change had on security.
% \end{theorem}

% \begin{theorem}\label{theo:intro:secmet_rate}
% If we test an aspect of cyber security at regular intervals, then we can determine the average rate of change for that security metric over the given time period.
% \end{theorem}

% Theorems \ref{theo:intro:secmet_diff} and \ref{theo:intro:secmet_rate} establish the basis for a \textit{cyber security calculus} that are a central premise of this work. 
