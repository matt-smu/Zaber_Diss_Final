
Aristotle is often credited with founding the fields of natural science in the West; with the story going that he disagreed with his teacher Plato on the matter of philosophical thought, arguing that it should be based on observations from the natural world. We consider observations to be  \textit{measurements} when they are quantified with respect to an agreed upon scale, or measurement unit. The first measurement units were used in bartering to standardize the exchange values of different commodities\cite{Morris_2001}. To measure length, for example, the torso was consistent enough across people to be considered a measurement unit, from which other units like the hand, the foot, and the cubit were derived. A \textit{quantity} refers specifically to a property that can be measured, in this case length. Although the term is often\cite{Debievre_2009} overloaded to mean the amount of something as well, we attempt to be metrologically consistent in this work by referring to the property to be measured as the quantity or \textit{measurand}, and the \textit{amount} of some thing as the observed value in terms of magnitude and measurement unit. 

In recent years effort has been spent investigating the science of cyber security\cite{Kott_2014, Schneider, Chang_2019, Spring_Moore_Pym_2017}. In the \textit{NSF/IARPA/NSA Workshop on the Science of Security}\cite{Evans_2008} three directions of research were identified to improve the scientific foundations of security, \textit{metrics}, \textit{formal methods}, and \textit{experimentation}. In this thesis we focus particularly on the role of metrics in security research, although well-defined system models and experimentation environments both play key roles in the analysis. In this chapter we review the relevant concepts in security metrics needed to build our framework. Section \ref{sec:background:modeling} describes methods for representing different components influencing security. These models are the inputs to the security property being measured, and in Section \ref{sec:background:metrics} we describe the types of metrics available and how they relate. %In section \ref{sec:background:metric_validation} we describe techniques used to ensure the instruments used to measure security are accurate and consistent. 