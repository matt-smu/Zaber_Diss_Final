This research examines the benefits and barriers to adopting Trusted Execution Environments (TEE) in DoD information systems. Specifically, we investigate applications of trusted enclaves at the client and provider edge in tactical network deployments. TEE integrity and confidentiality guarantees apply broadly to Information Dominance S\&T objectives. Since TEEs are implemented in modern commodity hardware distributed by vendors already in the DoD supply chain, taking advantage of these added assurances should not require additional hardware procurement. Like any other technology, efficacy claims must be weighed against evidence and considered within the context of the larger ecosystem, which this work begins to address. While we are aware of work within the command related to custom ASIC and FPGA cryptographic primitives, to our knowledge there is no other group researching these capabilities in commodity CPUs where they would find the widest adoption and lowest barrier to entry. 

This report addresses the NISE proposal objectives as outlined below:
\begin{itemize}
\item Section \ref{sec_sgx} attempts to enumerate and define security primitives of hardware TEEs, focusing on the Intel SGX implementation. 
\item Section \ref{sec_mec} reviews MEC network concepts relevant to our implementation.
\item Section \ref{subsec_storm} describes the HAEC proof of concept implementation, including hardware infrastructure and software implementation. 
\item Section \ref{subsec_rmf} maps NIST SP 800-53 inheritable security controls to TEE functionality.
\end{itemize}


