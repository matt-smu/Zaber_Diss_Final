
So far we have reviewed an assortment of models, which describe the syntax and grammar of the system components we wish to analyze, along with metrics, which capture some property of security we wish to quantify. To measure a specific security property a metric is applied to an appropriate collection of model instances. This mapping is usually implicit in the metric's definition - a vulnerability based metric assumes a vulnerability scanner has been run against a system and the results are packed in an appropriate format for the metric to operate on for example. Obtaining a measurement from input model instances may be trivial or could involve multiple pre-processing steps. We can think of the steps taken for a security measurement to be obtained as a processing pipeline for security analytics, and approach these steps like an extract, transform, load (ETL) pipeline in Chapter \ref{ch:automation}. For background we present in this section the analytics pipeline proposed in \cite{Abraham_2016}, as it forms the basis of the case-study in Section \ref{sec:case_studies:att}, and motivates much of the developed automation used in future chapters of this thesis.

%  Typically these security goals can be modeled in an attack graph using the attacker's intent and target objectives. We also take this opportunity to establish the notation used throughout the paper. Table \ref{tab:metric_summary} provides a brief summary of selected metrics.