

% The initial data model describing our input system is defined in the Multi-host, Multi-stage Vulnerability Analysis Language MulVal{\cite{Ou_Govindavajhala_Appel}}. MulVal is an efficient\cite{Rao_1997} Prolog based modeling language and inference engine for vulnerability relationship analysis. Information about the system under test is gathered through standard methods such as Nessus, nmap, or OVAL scans and parsed into the global system model. MulVal can then assert facts and  make inferences about how the entities are expected to interact and what illegal interactions are possible. Adding attacker origin and target elements to the model will yield a directed cyclic graph of all possible paths an attacker may traverse to compromise the target. The attack graph represents the exploitable vulnerabilities in a system as the set of connected nodes and edges between an attacker’s origin and the target. 

% This attack graph is exported as a CSV representation of the relationships between vulnerabilities on the network. This keeps the modeling and analysis phases loosely coupled and allows other tools to be easily integrated. The collection of vertices, edges, and attributes that are included in the attack graph output contain detailed information about the rules and facts that triggered the advance. 
