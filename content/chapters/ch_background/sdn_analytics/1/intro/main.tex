
This paper presents research focused on determining the security implications of proposed architectural changes during infrastructure migration. We consider a carrier network upgrading from existing core switching and routing elements to a centrally controlled Software Defined Network (SDN). We follow the process described in the Cyber Security Analytics Framework\cite{Abraham_Nair_2015a}to build metrics for 3 distinct network transition states, and use these metrics to evaluate the security posture of the proposed migration strategy.


% \subsection{Contributions}
In this work we develop a set of tools and processes that produce a holistic view of both an existing and proposed network’s security posture, allowing a user to model competing transition strategies or end states for direct comparison of the resulting security measurements. We extend the scope of \cite{Abraham_2016} from the enterprise to include core networking elements and associated metrics. Further, we implement this work in a modular framework to facilitate integration into existing monitoring and reporting systems. 


The remainder of the paper is structured as follows: Section \ref{sec:background} provides an overview of existing research and specific background topics related to this work. Section \ref{sec:contribs} describes this project's contributions to automating and extending previous research. Section \ref{sec:approach} establishes the use case and traces the flow of the application by example. Section \ref{sec:results} evaluates the resulting security measures and finally, we summarize our work in section \ref{sec:conclusion}.



