
An argument can be made that the accuracy of any measurement taken from a model is limited, and that the only true test of security is in measuring the production system. While it is possible to test systems in production, these tests can be disruptive to operations, for example when testing fire alarms in a building or conducting social engineering on customers or staff. We can recreate the production system on a cyber range\cite{Yamin_Katt_Gkioulos_2020} or similar dedicated environment for security testing, without the advantages of or impact on live traffic. In many cases we can interrogate the production system for values from which to build a model, but only if the system is in production. Design time tools are abundant throughout the CyBOK KAs, but the backend models used by tools differs in schema and accessibility, making integration with security metrics difficult. The role models play varies widely depending on the area of cyber security, with intrusion detection and antivirus systems on one end of the spectrum requiring production traffic for testing and evaluation, and cryptographic protocols on the other end having model based proofs of their security properties (although implementation is a different matter). The regulatory and compliance frameworks listed in Section \ref{sec:intro:threat_modeling} are made up of individual security controls, each of which can fall somewhere on this empirical spectrum. The use of models may be overly naive for some cases, but where applicable, models enable programmatic access and rapid prototyping, and allow us to isolate specific security properties for measurement through simulation and analysis at a speed not possible with live systems. In the rest of this section we provide some useful models that we make use of in later chapters. 