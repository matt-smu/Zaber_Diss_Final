% In this project, we identified distinct network models to represent point-in-time snapshots of the migration to an SDN controlled core network. The first model distills the existing architecture into a reduced set of interconnected network elements demonstrating the types of services and protocols in use today. The second model modifies the first by introducing centralized SDN control and supporting infrastructure while maintaining the logical topology and control plane services of the current architecture. Specifically, we maintain large ingress ACLs on both networks to prevent unauthorized traffic from reaching core infrastructure. The final network model represents the same SDN network infrastructure while implementing an MPLS overlay to isolate publicly addressable core infrastructure nodes from internet or customer edge originating attacks.  Vulnerabilities were then assigned to network elements in each of the models. These vulnerabilities are representative and are used to identify the types of exploits an attacker would make use of to advance through the network to a given target. When a new network model prevents an attacker from reaching the goal, no attack paths exist and no graph is generated. When this occurred, the outcome was noted, and hypothetical vulnerabilities were introduced in order to demonstrate comparison of the metrics provided by this framework. Attack graphs were then generated for each of the models and a script was developed to derive a stochastic model of the attack graph from the generated output. This model is the basis for calculating the set of metrics we used when comparing the security postures of the three reference network models. Our results demonstrate a novel application of the Cyber Security Analytics Framework by extending the existing enterprise network models and metrics to different carrier network architectures, and allowing direct comparison of multiple models based on the metrics we have derived. The use of attack graphs to capture relationships between vulnerabilities, both real and hypothetical, goes beyond looking at vulnerability data in isolation to provide a powerful means for advanced analysis of the entire network with a suite of metrics catered to our mission. Our findings for the EPL and NR measures show at a glance the improvements to network resiliency one model has over another, as well as critical points in the network that merit heightened attention from the IA team. 
In this paper we provided an end-to-end scenario demonstrating the application of the Cyber Security Analytics Framework to network migration planning. In this scenario we created distinct network models to represent point-in-time snapshots of the migration to an SDN controlled core network. The first model distills the existing architecture into a reduced set of interconnected network elements comprised of the types of services and protocols in use today. The second model modifies the first by introducing centralized SDN control and supporting infrastructure while maintaining the logical topology and control plane services of the current architecture. Specifically, we maintain large ingress ACLs on both networks to prevent unauthorized traffic from reaching core infrastructure. The final network model represents the same SDN network infrastructure while implementing an MPLS overlay to isolate publicly addressable core infrastructure nodes from internet or customer edge originating attacks.  

We developed a modular, automated implementation of the CSAF that is instrumented for customization. To streamline further research in this area, the following contributions were made:
\begin{itemize}
\item Infrastructure setup, provisioning, analysis, and reporting are implemented using industry standard open source tools, allowing testing to run locally or on the cloud with a single command. 
\end{itemize}
\begin{itemize}
\item Multiple end-to-end tests can be run in sequence or parallel as dictated by available resources.
\item Multiple network models can be specified for a test run while remaining logically organized and version controlled. 
\item Multiple custom rules sets can be applied individually or grouped to a test run, allowing results to reflect the subset of rules relevant to the analysis.
\item Vulnerabilities can now be weighted individually, by class, or by effect to facilitate 'what-if' analysis. In this context it is feasible to run a battery of tests against the provided architecture models in which vulnerabilities are applied stochastically and the security metrics are returned as heuristics.
\item  Transition matrix weighting strategies have been parameterized, allowing optimization or comparison of results. 
\item Metrics are extensible, customizable, and currently supported in \textbf{R} and \textbf{Python}. 
\end{itemize}

During the course of this project we came across some questions that are currently being investigated.  

When a network model prevents an attacker from reaching the goal, either because the model is truly secure or because no relevant vulnerability is defined in the rules or applied to the system, then no attack paths exist and subsequently no further analysis is conducted. When this occurred in our scenario above, the outcome was noted, and hypothetical vulnerabilities were introduced  to applicable models to allow comparison. In a non-planning (i.e., operational) environment, there must be a mechanism to delineate the true positive null results indicating an unreachable target from the false positive null results caused by a limited rule set. We are encouraged by the work presented in \cite{Stan_Bitton_Ezrets_Dadon_Inokuchi_Ohta_Yamada_Yagyu_Elovici_Shabtai_2019} as a means to bound this problem to the OSI layer. However, after working with our own network infrastructure attack rules and with those published by the research community, it became clear that  NVD entries don't always provide the information necessary to determine that an infrastructure attack is possible, or if it exists at all. While we can encode the conditions necessary to, for example, spoof an ARP response within a subnet, we are left without vetted CVSS exploitability and impact measures if the exploit isn't tied to a specific piece of software. In this work we provide the user a mechanism to define these scores for a general vulnerability class or for a single instance of that vulnerability, so using the CVSS calculator with a knowledge of the system under test should yield reasonable estimates. As part of ongoing research we also allow for custom weighting strategies for cases where CVSS is not applicable.

The use of attack graphs to capture relationships between vulnerabilities, both real and hypothetical, goes beyond looking at vulnerability data in isolation to provide a powerful means for advanced analysis of the entire network with a suite of metrics catered to our mission.


 

% As metrics are integrated and models are refined, a measurable account of the network security posture will emerge, creating a feedback loop where architects and security engineers can identify areas of risk, implement mitigations, and make adjustments to the model to determine the impact. Going forward, we would like to expand the set of available metrics to include the full suite identified in [1] as well as develop new metrics based on feedback and further research. We would also like to create a model visualization application capable of translating between the attack graph generation tool’s syntax and a common representation such as UML. Finally, the project would also benefit from further effort in increasing automation across the different phases, particularly in integrating existing tools into the model generation phase. These expanded features would support faster iterations, more realistic models, and enhanced value to the information assurance stakeholders.  