% \section{Experimental Results}\label{subsec:results:main}
% 
% \chapter{Conclusions \& Timeline} \label{ch:conclusion}

\input{content/chapters/ch_conclusion/conclusion_intro}


\section{Conclusions \& Future Work}\label{sec:conclusion:conclusion}


System security metrics are valuable only if they can produce timely, actionable measurements. In this thesis we have demonstrated a path forward for developing, testing, validating, and integrating security metrics into the full life cycle of a system. 

In Chapter \ref{ch:background} we present the current state of security metrics. We list the working taxonomies that these metrics can be categorized by, and elaborate on the distinctions that lead to confusion when discussing security measurements. We then review modeling techniques and how these models can isolate the security properties of a system we intend to measure. 

Chapter \ref{ch:automation} presents our unified framework for security measurement and analysis, including the model for implementing individual metrics and the infrastructure built around these metrics to drive automation in a variety of scenarios. Here we establish the security metric inheritance hierarchy and enumerate properties common to all metric types and those specific to each metric subtype. We provide our extensions to attack models that expand the range of systems that can be represented. We describe how we implemented automation from the view points of a security researcher or measurement analyst, and develop our concept of security metrics as a service, \textit{S-MaaS}, with considerations for deployment in a continuous integration or stream processing environment.

Chapter \ref{ch:benchmarking} develops our solution to the lack of validation in the field of security metrics. We establish a set of validation criteria that are needed for acceptance of any metric. We define a fixed set of models that set a frame of reference for evaluating security metrics, and explain how these models can be used to isolate key properties of interest. We investigate the instrumentation needed to validate our metrics in a general manner, and implement this validation framework as extensions to an industry accepted benchmarking tool to maximize the audience and reduce friction to entry. Finally we demonstrate our enhancements built around benchmarking to automate the process of executing tests, analyzing results, and alerting on anomalies and outliers that are uncovered during large scale or long running tests.

Chapter \ref{ch:case_studies} presents a case study conducted as part of AT\&T's planning for infrastructure migration. The study applies the CSAF\cite{Abraham_Nair_2015a} pipeline to hypothetical network architectures and gives insight into how model based security metrics can be used to rank an analysis of competing alternatives. As this study occurred early in the research phase, it had a great impact on the direction this thesis has taken. With the benefit of hindsight we are able to demonstrate both the contributions made during that initial work as well as the progress that has been made since it was completed. 



\begin{figure}[ht]
\centering
\includegraphics[width=\textwidth]{resource/img/ch_future/timeline_broad.png}
\caption{General Progression and Direction of Thesis}
\label{fig:future:timeline_broad}
\end{figure} 

Figure \ref{fig:future:timeline_broad} captures with broad strokes the path our research has followed and the direction it is heading, while the timeline in Figure \ref{fig:future:timeline_detail} summarizes previous research items that support this thesis. Listed along the top are the two long running projects that have provided both requirements and solutions in this work. SDN Migration Analytics is the focus of the case study in Chapter \ref{ch:case_studies} while the Cloud Benchmarking research has provided an in depth knowledge of designing and validating cyber measurement instruments. Immediately below the long running projects are short term studies conducted over summers each year. The Tactical Edge work that bookends the summer research items focused on evaluating the security of non traditional network architectures and drove our requirement to validate metrics outside of the enterprise domains commonly found in the literature. The Maru research over the summer of 2017 and 2018 led to the development of a distributed streaming analytics system run from within a hardware trusted enclave, which forms the basis of the S-MaaS architecture described in \ref{sec:smaas:arch}. On the right side a legend designates presentations, posters, and papers delivered that relate to the research topics listed above.


\begin{figure}[ht]
\centering
\includegraphics[width=\textwidth]{resource/img/ch_future/timeline.png}
\caption{Timeline of Work Supporting Thesis }
\label{fig:future:timeline_detail}
\end{figure} 

% The timeline in Figure \ref{fig:future:furure_work} identifies critical degree requirement milestones along the top and anticipated deliverables along the bottom. 

% Currently we are preparing submissions, adding models and collecting evidence for the metric validation work, and plan to submit that around the same time as qualifying exams. We are able to create the labeled datasets for the ML models as part of the validation work, and can begin training once the datasets are created. As we discussed in Chapter \ref{ch:future}, the applications of ML in cyber security are limited in scope to a small set of applications like traffic classification for intrusion detection and flow analysis for DDoS prevention. Nothing we found in the review of the literature considered system or threat topology when applying ML.a  We feel confident that our work creates novel datasets that will lead to interesting and valuable results in applied learning techniques. 



% Before drawing any conclusions from the analysis, it should be reiterated that the results obtained rely on representative threat estimates. Refinement and feedback from an Information Assurance (IA) Officer familiar with these network architectures are needed to ensure the accuracy of the models.    

The results in this section demonstrate how comparison between architectures can be accomplished empirically using the security metrics presented in Section \ref{ch:background}. Initial parameters did not produce attack graphs for the final state model. The implication is that this architecture was 'secure' given the identified vulnerabilities, attacker origin, and target. 

To continue our analysis we conducted 'what-if' testing by introducing hypothetical vulnerabilities into the models to represent as yet unknown attacks against specific infrastructure services and devices. Testing automation allowed us to run and collect results from 20 competing models during this stage of the analysis. 


\subsubsection{Structural Metrics}\label{subsec:results:sp}
% Our findings from the structural algorithms for the three network models under test can be found in Table \ref{tab:sp_results}. The shortest path metric is a measure of the path of least resistance from an attacker's origin to the target, and can be considered a priority when identifying risk in a network. %Theoretically, it could also be used by an attacker to identify the most direct route to a target. Another consideration is that an attacker may want to determine shortest paths as part of a minimal cut set algorithm for efficiently intercepting or degrading the target’s communications. 

\begin{table}[H]
\caption{Structural Metric Results Summary}
\begin{tabular}{@{}lllll@{}}
\toprule
Structural Path Metric & Current & Transition & Final &  \\ \midrule
Shortest Path (SP) & 4 & 3 & 3 &  \\
Number of Paths (NP) & 6 & 3 & 1 &  \\
Mean Path Length (MPL) & 5.33 & 4 & 3 &  \\ \bottomrule
\end{tabular}
\label{tab:sp_results}
\end{table}

Our findings from the structural algorithms for the three network models under test can be found in Table \ref{tab:sp_results}. The shortest path metric is a measure of the path of least resistance from an attacker's origin to the target, and can be considered a priority when identifying risk in a network. %Theoretically, it could also be used by an attacker to identify the most direct route to a target. Another consideration is that an attacker may want to determine shortest paths as part of a minimal cut set algorithm for efficiently intercepting or degrading the target’s communications. 

\begin{table}[ht]
\caption{Structural Metric Results Summary}
\begin{tabular}{@{}lllll@{}}
\toprule
Structural Path Metric & Current & Transition & Final &  \\ \midrule
Shortest Path (SP) & 4 & 3 & 3 &  \\
Number of Paths (NP) & 6 & 3 & 1 &  \\
Mean Path Length (MPL) & 5.33 & 4 & 3 &  \\ \bottomrule
\end{tabular}
\label{tab:sp_results}
\end{table}

% \subsection{Node Rank}\label{subsec:results:nr}
% 
\textbf{Node Ranking (NR):  }
Figure \ref{fig:ag_all} shows the reduced attack graphs for the given models and Figure \ref{fig:nr_all} provides the corresponding Node ranks. Remember that node rank is a measure of the amount of time we expect an attacker to spend before succeeding at a given exploit, so higher values here are preferable for from a defender's point of view. We see in Figure \ref{fig:nra_fin} that  node X11 will take much longer to exploit than any other vulnerability. 
% We have shown that the elements of the fundamental matrix \(F\) take on values that represent the relative duration of time spent at each transient node in the Markov process. In the context of our security analysis, these values equate to the amount of hold time we expect an attacker to incur while trying to advance to the target. Lower node rankings indicate nodes along the attack path that are relatively easy for an attacker to clear. If a difficult to exploit vulnerability exists and its associated NR is relatively low this might be an indication that a security control point is being bypassed. Using the attack graph and associated NR analysis it is a fairly straight forward process to identify the area of interest and trace back to the origin of the bypass. Liu\cite{Liu_Singhal_Wijesekera} examines this process of forensic reconstruction of attacks using attack graphs in detail.



\begin{figure*}[ht]
\centering
\begin{adjustbox}{minipage=\linewidth,scale=.8}
\begin{subfigure}{.33\textwidth}
%\includegraphics[width=\linewidth,height=6cm]{img/1553187466086.png}
% \includegraphics[width=\linewidth,height=6cm]{content/figs/net_ags_003.png}
\includegraphics[width=\linewidth,height=6cm]{content/figs/weightedGraphs/current_007_weighEdges.png}
\caption{current}
\label{fig:ag_currt}
\end{subfigure}%
\begin{subfigure}{.33\textwidth}
%%\includegraphics[width=\linewidth,height=6cm]{img/1553187466087.png}
% \includegraphics[width=\linewidth,height=6cm]{content/figs/net_ags_002.png}
\includegraphics[width=\linewidth,height=6cm]{content/figs/weightedGraphs/transition_007_weighEdges.png}
\caption{transition}
\label{fig:ag_trans}
\end{subfigure}%
\begin{subfigure}{.2\textwidth}
%\includegraphics[width=\linewidth,height=6cm]{img/1553187466083.png}
% \includegraphics[width=\linewidth,height=5cm]{content/figs/net_ags_001.png}
\includegraphics[width=\linewidth,height=5cm]{content/figs/weightedGraphs/final_007_weighEdges.png}
\caption{final}
\label{fig:ag_fut}
\end{subfigure}%
\caption{Generated Attack Graphs}
\label{fig:ag_all}
\end{adjustbox}
\end{figure*} 


\begin{figure*}[ht]
\centering
\begin{adjustbox}{minipage=\linewidth,scale=1}
\begin{subfigure}{.33\textwidth}
\includegraphics[width=\linewidth]{img/1553187466081.png}
\caption{current}
\label{fig:nra_curr}
\end{subfigure}%
\begin{subfigure}{.33\textwidth}
\includegraphics[width=\linewidth]{img/1553187466082.png}
\caption{transition}
\label{fig:nra_trans}
\end{subfigure}%
\begin{subfigure}{.33\textwidth}
\includegraphics[width=\linewidth]{img/epl_final.png}
\caption{final}
\label{fig:nra_fin}
\end{subfigure}%
\caption{Node Rank Analysis}
\label{fig:nr_all}
\end{adjustbox}
\end{figure*} 



\textbf{Node Ranking (NR):  }

\begin{figure}[H]
\centering
% \begin{adjustbox}{minipage=\linewidth,scale=0.6}
\begin{subfigure}{.33\textwidth}
%\includegraphics[width=\linewidth,height=6cm]{img/1553187466086.png}
\includegraphics[width=\linewidth,height=6cm]{content/chapters/ch_background/sdn_analytics/2/figs/net_ags_003.png}
\caption{current}
\label{fig:ag_currt}
\end{subfigure}%
\begin{subfigure}{.33\textwidth}
%%\includegraphics[width=\linewidth,height=6cm]{img/1553187466087.png}
\includegraphics[width=\linewidth,height=6cm]{content/chapters/ch_background/sdn_analytics/2/figs/net_ags_002.png}
\caption{transition}
\label{fig:ag_trans}
\end{subfigure}%
\begin{subfigure}{.2\textwidth}
%\includegraphics[width=\linewidth,height=6cm]{img/1553187466083.png}
\includegraphics[width=\linewidth,height=5cm]{content/chapters/ch_background/sdn_analytics/2/figs/net_ags_001.png}
\caption{final}
\label{fig:ag_fut}
\end{subfigure}%
\caption{Generated Attack Graphs}
% \end{adjustbox}
\end{figure} 


\begin{figure}[H]
\centering
% \begin{adjustbox}{minipage=\linewidth,scale=0.6}
\begin{subfigure}{.33\textwidth}
\includegraphics[width=\linewidth]{content/chapters/ch_background/sdn_analytics/2/figs/weightedGraphs/current_007_weighEdges.png}
\caption{current}
\label{fig:nra_curr}
\end{subfigure}%
\begin{subfigure}{.33\textwidth}
\includegraphics[width=\linewidth]{content/chapters/ch_background/sdn_analytics/2/figs/weightedGraphs/transition_007_weighEdges.png}
\caption{transition}
\label{fig:nra_trans}
\end{subfigure}%
\begin{subfigure}{.33\textwidth}
\includegraphics[width=\linewidth]{content/chapters/ch_background/sdn_analytics/2/figs/weightedGraphs/final_007_weighEdges.png}
\caption{transition}
\label{fig:nra_fin}
\end{subfigure}%
\caption{Node Rank Analysis}
% \end{adjustbox}
\end{figure} 

\subsubsection{Expected Path Length}\label{subsec:results:epl}
% 
\textbf{Expected Path Length (EPL):  }

% We define EPL as the expected number of time steps required for an attacker to advance from the initial state to the attack goal, and its calculation follows as a direct consequence of deriving the NR metric. That is, if the NR metric expresses the total expected time that a process starting in initial state $s_i$ will occur in transient state \(s_j\) before ultimately being absorbed, then the NR sum over all transient states for \(s_i\) will predict the total time spent in the process before absorption. To take the sum of the values in the rows of the fundamental matrix we multiply by a column of 1’s, t = N1, and the entry \(t_i\) contains the EPL value for initial state \(s_i\). 
EPL describes how long we can expect an attacker to be in our network before the target is successfully compromised. Table \ref{tab:epl_result} shows the EPL values for each model along with the path length histogram of the simulations that were run. This histogram counts how many times the simulation reached the target in exactly Path Length steps. We can infer that the higher the EPL value, the longer an attacker will attempt to advance to the target, and the more chance we have to observe and act in response [1]. The long tail on the final state histogram indicates that, in several simulations, the observed path length nearly tripled that of the other two models. We notice that, despite having significantly more available attack paths (NP(current)=6 vs NP(SDN)=3) the expected path length of the current model is actually higher than that of the SDN model. Likewise, although the SDN models  both have Shortest Path scores = 3, we have shown the resiliency of these networks to be unequal.  In doing so we demonstrate the additional insight provided by incorporating vulnerability awareness into our threat modelling and planning tools.

% \begin{figure*}[ht]
\begin{table*}[ht]
\caption{Expected Path Length Results}
\resizebox{\textwidth}{!}{%
\begin{tabular}{@{}llll@{}}
\toprule
Current & Transition & Final &  \\ \midrule
Expected Length: 9.398 & Expected Length: 8.0875 & Expected Length: 11.1405 &  \\ \bottomrule
\raisebox{-\totalheight}{\includegraphics[width=0.3\textwidth, height=60mm]{img/pathlength_curr.png}}
      & 
\raisebox{-\totalheight}{\includegraphics[width=0.3\textwidth, height=60mm]{img/pathlength_trans.png}}
&
\raisebox{-\totalheight}{\includegraphics[width=0.3\textwidth, height=60mm]{img/pathlength_final.png}}
\\
\end{tabular}
}
\label{tab:epl_result}
\end{table*}

% \end{figure*}

 


\textbf{Expected Path Length (EPL):  }

% We define EPL as the expected number of time steps required for an attacker to advance from the initial state to the attack goal, and its calculation follows as a direct consequence of deriving the NR metric. That is, if the NR metric expresses the total expected time that a process starting in initial state $s_i$ will occur in transient state \(s_j\) before ultimately being absorbed, then the NR sum over all transient states for \(s_i\) will predict the total time spent in the process before absorption. To take the sum of the values in the rows of the fundamental matrix we multiply by a column of 1’s, t = N1, and the entry \(t_i\) contains the EPL value for initial state \(s_i\). 
EPL describes how long we can expect an attacker to be in our network before the target is successfully compromised. Table \ref{tab:epl_result} shows the EPL values for each model along with the path length histogram of the simulations that were run. This histogram counts how many times the simulation reached the target in exactly Path Length steps. We can infer that the higher the EPL value, the longer an attacker will attempt to advance to the target, and the more chance we have to observe and act in response [1]. The long tail on the final state histogram indicates that, in several simulations, the observed path length nearly tripled that of the other two models. We notice that, despite having significantly more available attack paths (NP(current)=6 vs NP(SDN)=3) the expected path length of the current model is actually higher than that of the SDN model. Likewise, although the SDN models  both have Shortest Path scores = 3, we have shown the resiliency of these networks to be unequal.  In doing so we demonstrate the additional insight provided by incorporating vulnerability awareness into our threat modelling and planning tools.

\begin{table}[H]
\caption{Expected Path Length Results}
\begin{tabular}{@{}llll@{}}
\toprule
Current & Transition & Final &  \\ \midrule
Expected Length: 9.398 & Expected Length: 8.0875 & Expected Length: 11.1405 &  \\ \bottomrule
\raisebox{-\totalheight}{\includegraphics[width=0.3\textwidth, height=60mm]{resource/img/ch_casestudies/att/pathlength_curr.png}}
      & 
\raisebox{-\totalheight}{\includegraphics[width=0.3\textwidth, height=60mm]{resource/img/ch_casestudies/att/pathlength_trans.png}}
&
\raisebox{-\totalheight}{\includegraphics[width=0.3\textwidth, height=60mm]{resource/img/ch_casestudies/att/pathlength_final.png}}
\\
\end{tabular}
\label{tab:epl_result}
\end{table}


 

\subsubsection{Probabilistic Path}\label{subsec:results:pp}
% 
\(
\textbf{Probabilistic Path (PP):} 
\)
% The PP metric is another interesting property derived from our Markov transition matrix. Taking the product of the fundamental matrix F and the matrix of absorbing probabilities \(R\) results in a matrix, \(B = FR\), whose \(B(i,j) \) entries yield the probability of being absorbed by state \(s_j\) given we started at initial state \(s_i\).  


\begin{figure}[ht]
\centering
\includegraphics[width=.48\textwidth]{img/ag_extra_target.png}
\caption{Transition diagram with additional absorbing states New\_Target}
\label{fig:ag_pp}
\end{figure}


Recall that entries in the PP matrix \(B = FR\) produce the probability of being absorbed by state \(s_j\) from an initial state \(s_i\).  In the current scenarios we define a single attack target, so calculating this PP metric results in a column of 1’s due to absorbing markov chains requiring 100\% chance of absorption from any transient node by definition.   

However it is easy to imagine a case where multiple targets are specified in the model. For example, we have added a second absorbing state, ‘New\_Target’ to the transition diagram from Figure \ref{fig:ag_pp}. This could represent a hot failover clone of the existing Target or it could be a completely separate system with unique vulnerabilities. In either case we can make a grounded prediction on which state will absorb the attacker with the highest likelihood and prepare accordingly. 




\(
\textbf{Probabilistic Path (PP):} 
\)
% The PP metric is another interesting property derived from our Markov transition matrix. Taking the product of the fundamental matrix F and the matrix of absorbing probabilities \(R\) results in a matrix, \(B = FR\), whose \(B(i,j) \) entries yield the probability of being absorbed by state \(s_j\) given we started at initial state \(s_i\).  


\begin{figure}[H]
\centering
\includegraphics[width=100mm]{resource/img/ch_casestudies/att/ag_extra_target.png}
\caption{Transition diagram with additional absorbing states New\_Target}
\label{fig:ag_pp}
\end{figure}


In the current scenario calculating this PP metric results in a column of 1’s since we only define a single ‘Target’ node in each of our models with 100\% chance of absorption from any transient node by definition.   

However it is easy to imagine a case where multiple targets are specified in the model. For example, we have added a second absorbing state, ‘New\_Target’ to the transition diagram from Figure \ref{fig:ag_2}. This could represent a hot failover clone of the existing Target or it could be a completely separate system with unique vulnerabilities. In either case we can make a grounded prediction on which state will absorb the attacker with the highest likelihood and prepare accordingly. 



 



