

A model is a simplified representation of some entity. As with security metrics in Section \ref{sec:intro:sys_sec}, security models can be defined and applied in different ways across the various areas of cyber security. Formal methods are a widely used technique in computer science to specify a hardware or software system as a mathematical model and verify its behaviour\cite{Bell_LaPadula_1973}. Attack and defense modeling go hand in hand in many\cite{Duggan_Michalski, Ellison, Hutchins_Cloppert_Amin, Morana_2015, Schneier_1999, Schoenfield, Shostack, Woodard_Veitch_Thomas_Duggan_2007} threat modeling frameworks. Cyber ranges\cite{Costa_Russo_Armando} are partially or fully functioning replicas of existing cyber systems built out to test attack or defense capabilities.  Of particular interest in this work is Mitre's various\cite{Corporation} data models and enumerations of Common Weaknesses (CWE), Attack Patterns (CAPEC), Malware Attributes (MAEC), and APT group characterizations (ATT\&CK) which we describe in detail later in this work. 

% Modeling a system's vulnerabilities and the reachability between those vulnerabilities can be found in the literature as far back as 1994 with Dacier\cite{Dacier_1994} formalising the concept of privilege graphs and representing the translated graph as a Markov Model. Phillips and Swiler\cite{Phillips_Swiler_1998} present a separate attack graph generation method that can account for multi-stage attacks and attacker capabilities in 1998. In 1999 Ortalo\cite{Ortalo_1999}  provides experimental results and some fundamental metrics using Markov analysis with Dacier's privilege graphs, and in 2002 Sheyner\cite{Sheyner_Haines_Jha_Lippmann_Wing_2002} describes how attack graph construction and analysis can be automated. In 2006 Ou\cite{Ou_Boyer_McQueen_2006} provides an analysis of scalability extensions to the MulVal\cite{Ou_Govindavajhala_Appel} attack graph engine presented the previous year, and in 2013 Hong\cite{Hong_Kim_Takaoka_2013} presents further scalability improvements to MulVal using logic reduction techniques. 
% In 2015 Abraham\cite{Abraham_Nair_2015b} introduces the Cyber Security Analytics Framework(CSAF) which we adopt for this analysis. More thorough surveys of the canonical attack graph literature can be found in \cite{Kordy_2013} and \cite{Lippmann_Ingols_2005}. We use the remainder of this section to illustrate the MulVal inputs and outputs as expected by the CSAF. 

% Attack graphs show the relationships among vulnerabilities within a system and provide context to security scans already conducted by many organisations. An attack graph is a directed graph that captures all possible paths an attacker can traverse within a system to reach a desired target state. The first node in the graph represents the origin of the attack and the final node denotes the target. The origin contains only outbound edges and the target contains only inbound edges. Nodes in the graph between the origin and target represent discrete states in states network. Each edge in the graph identifies a possible pivot from one state to another through either unaltered access mechanisms or successful exploitation of a vulnerability. The conditions necessary for successful compromise of the vulnerability are encapsulated in the attack graph vertices, and include information such as network, port, protocol, and access privilege level restrictions, as well as the effect of a successful exploit on the system such as privilege escalation or remote code execution. These conditions can be populated from the output of IA and network management systems, or in hypothetical cases, can be defined manually.  