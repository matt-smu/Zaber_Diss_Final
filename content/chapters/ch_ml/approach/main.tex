


As discussed in the literature review, the majority of machine learning applications in the cyber security realm have been focused on intrusion detection techniques. Rather than revisit those efforts, through our previous work we are in a unique position to take advantage of existing measures of security to assist in training and optimization of machine learning algorithms. While there are many security metrics in the literature, we focus primarily on those that already rely on attack graphs to calculate a metric. Pendleton’s survey\cite{Pendleton_Garcia-Lebron_Cho_Xu_2016} focuses on metrics that quantify attack and defense interactions, and these are the broadly descriptive measurements that would provide the most benefit to operators in the field. These metrics are classified as measuring one or more of Vulnerabilities, Threats, Defenses, or Situations. Situations in this case is a comprehensive metric, with Pendleton’s example subgroups measuring security state over time, successful attacks over time (incident rate), and return on economic investment. Again we find a subjectivity in the  assignment of users’ susceptibility and attack and defense effectiveness scores. Even CVSS is bound to variance for any vulnerabilities not pre assigned a score. In comparison, analogous system wide performance metrics like those found in workload simulations (YCSB for example) will be fairly deterministic. Verendel\cite{Verendel_2009} makes a critical analysis of the claim that security is quantifiable. The premise is that most of the published models and metrics that attempt to measure security lack the scientific rigor to corroborate or validate their hypothesis. The scope of is limited to operational security measurements and assumes measurement primitives include systems, threats, and vulnerabilities.

Our methodology then is to explore areas of machine learning that have have not yet been available to the cyber security community. To this end, we have a variety of options available, not all of which will be productive. Our first approach is to label a set of system models with the value assigned by respective security metrics. In this way we create a labeled training set that can be used to train a model which avoids the expensive overhead of evaluating metrics continuously through analytical methods. Given the nature of many stochastic metrics, there is also a great potential for applying unsupervised learning methods to our data set, which we will also describe in the next sections. 