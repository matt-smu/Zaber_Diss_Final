


In this chapter we investigate how machine learning can be used in conjunction with the metrics and measurements we are surfacing to improve our understanding of cyber security. 

% Specifically, we focus on the following 3 areas in this paper:
 
% \begin{enumerate}
% \item \textbf{Learning valid security metrics}: Can we apply graph clustering and similarity methods to group models by valid cyber security properties?
% \item \textbf{Model Scoring for Network Design Support}: Can we make use of existing system and threat models, along with associated security metrics, to better model attacker behavior and improve incident response? 

% \item \textbf{Vulnerability Score Fuzzing to predict metric}: Given a network model, can we predict the values of specific security metrics through classification or regression? 
% \end{enumerate}


In the December 2019 workshop\textit{ Implications of Artificial Intelligence for Cybersecurity}\cite{Chang_2019}, one of the key takeaways identified was the need to expand the connections between cyber security and applications of artificial intelligence and machine learning. In this work we have so far focused on making cyber security measurable, focusing on instrumentation for automation and autonomy. Programmatic access to security metrics through automation opens up a wide variety of applications involving, and can itself be improved by, current techniques in machine learning. In this chapter we describe the design details for the experiments we are conducting using the SMaaS environment. While adversarial AI and attacks on ML models are areas of concern today, the direction of this work is \textit{not} in applying SMaaS to evaluate the security of machine learning. Instead, we propose to investigate the use of specific machine learning techniques to improve cyber security through the metrics framework we developed above. 
