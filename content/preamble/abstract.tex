Model based security metrics are a growing area of cyber security research concerned with measuring the risk exposure of an information system. These metrics are typically studied in isolation, with the formulation of the test itself being the primary finding in publications. As a result, there is a flood of metric specifications available in the literature but a corresponding dearth of analyses verifying results for a given metric calculation under different conditions or comparing the efficacy of one measurement technique over another. The motivation of this thesis is to create a systematic methodology for model based security metric development, analysis, integration, and validation. In doing so we hope to fill a critical gap in the way we view and improve a system’s security.   

In order to understand the security posture of a system before it is rolled out and as it evolves, we present in this dissertation an end to end solution for the automated measurement of security metrics needed to identify risk early and accurately. To our knowledge this is a novel capability in design time security analysis which provides the foundation for ongoing research into predictive cyber security analytics. Modern development environments contain a wealth of information in infrastructure-as-code repositories, continuous build systems, and container descriptions that could inform security models, but risk evaluation based on these sources is ad-hoc at best, and often simply left until deployment. Our goal in this work is to lay the groundwork for security measurement to be a practical part of the system design, development, and integration lifecycle.  

In this thesis we provide a framework for the systematic validation of the existing security metrics body of knowledge. In doing so we endeavour not only to survey the current state of the art, but to create a common platform for future research in the area to be conducted. 

We  then  demonstrate  the  utility  of  our  framework  through  the  evaluation  of  leading security metrics against a reference set of system models we have created.  We investigate how  to  calibrate  security  metrics  for  different  use  cases  and  establish  a new methodology  for security metric benchmarking. 

We further explore the research avenues unlocked by automation through our concept of an API driven S-MaaS (Security Metrics-as-a-Service) offering. We review our design considerations in packaging security metrics for programmatic access, and discuss how various client access-patterns are anticipated in our implementation strategy. Using existing metric processing pipelines as reference, we show how the simple, modular interfaces in S-MaaS support dynamic composition and orchestration. 

Next we review aspects of our framework which can benefit from optimization and further automation through machine learning. First we create a dataset of network models labeled with the corresponding security metrics. By training classifiers to predict security values based only on network inputs, we can avoid the computationally expensive attack graph generation steps. We use our findings  from this simple experiment to motivate our current lines of research into supervised and unsupervised techniques such as network embeddings, interaction rule synthesis, and reinforcement learning environments. 

Finally, we examine the results of our case studies. We summarize our security analysis of a large scale network migration, and list the friction points along the way which are remediated by this work. We relate how our research for a  large-scale performance benchmarking project has influenced our vision for the future of security metrics collection and analysis through dev-ops automation. We then describe how we applied our framework to measure the incremental security impact of running a distributed stream processing system inside a hardware trusted execution environment.

% To summarize:
% We created a rapid prototyping and integration environment for security metrics. 
% We developed a reference data set for validating security metrics across different topologies and scales.
% We implemented benchmark tests in a widely adopted open source benchmark test suite to reach the largest audience. 
% We developed S-MaaS, a scalable deployment system where metrics are microservices.
% We enhanced S-MaaS automation using a variety of machine learning applications.
% We present case studies and lessons learned as evidence our framework is practical and effective.
