
Observation is the foundation of scientific experimentation. We consider observations to be measurements when they are quantified with respect to an agreed upon scale, or measurement unit. A number of metrics have been proposed in the literature which attempt to quantify some property of cyber security, but no systematic validation has been conducted to characterize the behaviour of these metrics as measurement instruments, or to understand how the quantity being measured is related to the security of the system under test. In this paper we broadly classify the body of available security metrics against the recently released Cyber Security Body of Knowledge, and identify common attributes across metric classes which may be useful anchors for comparison. We propose a general four stage evaluation pipeline to encapsulate the processing specifics of each metric, encouraging a separation of the actual measurement logic from the model it is often paired with in publication. Decoupling these stages allows us to systematically apply a range of inputs to a set of metrics, and we demonstrate some important results in our proof of concept. First, we determine a metric's suitability for use as a measurement instrument against validation criteria like operational range, sensitivity, and precision by observing performance over controlled variations of a reference input. Then we show how evaluating multiple metrics against common reference sets allows direct comparison of results and identification of patterns in measurement performance. Consequently, development and operations teams can also use this strategy to evaluate security tradeoffs between competing input designs or to measure the effects of incremental changes during production deployments. 


% Steps to organizing your manuscript
% \begin{enumerate}
% \item Prepare the figures and tables
% \item Write the Methods.
% \item  Write up the Results.
% \item Write the Discussion. Finalize the Results and Discussion before writing the introduction. This is because, if the discussion is insufficient, how can you objectively demonstrate the scientific significance of your work in the introduction?
% \item Write a clear Conclusion.
% \item   Write a compelling introduction.
% \item Write the Abstract.
% \item Compose a concise and descriptive Title.
% \item  Select Keywords for indexing.
% \item Write the Acknowledgements.
% \item Write up the References.
% \end{enumerate}


% The abstract tells prospective readers what you did and what the important findings in your research were. Together with the title, it's the advertisement of your article. Make it interesting and easily understood without reading the whole article.  Avoid using jargon, uncommon abbreviations and references.

% You must be accurate, using the words that convey the precise meaning of your research. The abstract provides a short description of the perspective and purpose of your paper. It gives key results but minimizes experimental details. It is very important to remind that the abstract offers a short description of the interpretation/conclusion in the last sentence.

% A clear abstract will strongly influence whether or not your work is further considered.

% However, the abstracts must be keep as brief as possible. Just check the 'Guide for authors' of the journal, but normally they have less than 250 words. Here's a good example on a short abstract.

% In an abstract, the two whats are essential. Here's an example from an article I co-authored in Ecological Indicators:

%     What has been done? "In recent years, several benthic biotic indices have been proposed to be used as ecological indicators in estuarine and coastal waters. One such indicator, the AMBI (AZTI Marine Biotic Index), was designed to establish the ecological quality of European coasts. The AMBI has been used also for the determination of the ecological quality status within the context of the European Water Framework Directive. In this contribution, 38 different applications including six new case studies (hypoxia processes, sand extraction, oil platform impacts, engineering works, dredging and fish aquaculture) are presented."
%     What are the main findings? "The results show the response of the benthic communities to different disturbance sources in a simple way. Those communities act as ecological indicators of the 'health' of the system, indicating clearly the gradient associated with the disturbance."

    
    
   
    
    
  
    
    
   
    
    
