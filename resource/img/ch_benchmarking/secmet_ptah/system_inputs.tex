


In the \textit{Input} stage of Figure \ref{fig:automation:metric_pipeline}, system details depicted above the inbound arrows on the left are parsed into a model which describes the current environment or environment under test. Model parameters can be populated synthetically or from a live system. Rules comprising the threat model which describes how these components are allowed to interact with each other and with external stimulus can be added here for metrics that require it. The raw inputs to the processing pipeline can vary widely depending on which security metrics are being considered. At a high level, we treat the input stage as a black box for handling information requests from subsequent stages. This affords us the freedom to connect static data for testing and experimentation, and live data for production deployments without altering the contract or interface. In practice input targets can be existing APIs provided by SIEMs, query interfaces to a configuration database, source code repositories, vulnerability information feeds, generated network topologies, etc. At this stage we only assume appropriate adapters exist to make this data available for the subsequent stages.
