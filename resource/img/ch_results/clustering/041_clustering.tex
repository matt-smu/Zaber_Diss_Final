\documentclass[11pt]{article}

    \usepackage[breakable]{tcolorbox}
    \usepackage{parskip} % Stop auto-indenting (to mimic markdown behaviour)
    
    \usepackage{iftex}
    \ifPDFTeX
    	\usepackage[T1]{fontenc}
    	\usepackage{mathpazo}
    \else
    	\usepackage{fontspec}
    \fi

    % Basic figure setup, for now with no caption control since it's done
    % automatically by Pandoc (which extracts ![](path) syntax from Markdown).
    \usepackage{graphicx}
    % Maintain compatibility with old templates. Remove in nbconvert 6.0
    \let\Oldincludegraphics\includegraphics
    % Ensure that by default, figures have no caption (until we provide a
    % proper Figure object with a Caption API and a way to capture that
    % in the conversion process - todo).
    \usepackage{caption}
    \DeclareCaptionFormat{nocaption}{}
    \captionsetup{format=nocaption,aboveskip=0pt,belowskip=0pt}

    \usepackage[Export]{adjustbox} % Used to constrain images to a maximum size
    \adjustboxset{max size={0.9\linewidth}{0.9\paperheight}}
    \usepackage{float}
    \floatplacement{figure}{H} % forces figures to be placed at the correct location
    \usepackage{xcolor} % Allow colors to be defined
    \usepackage{enumerate} % Needed for markdown enumerations to work
    \usepackage{geometry} % Used to adjust the document margins
    \usepackage{amsmath} % Equations
    \usepackage{amssymb} % Equations
    \usepackage{textcomp} % defines textquotesingle
    % Hack from http://tex.stackexchange.com/a/47451/13684:
    \AtBeginDocument{%
        \def\PYZsq{\textquotesingle}% Upright quotes in Pygmentized code
    }
    \usepackage{upquote} % Upright quotes for verbatim code
    \usepackage{eurosym} % defines \euro
    \usepackage[mathletters]{ucs} % Extended unicode (utf-8) support
    \usepackage{fancyvrb} % verbatim replacement that allows latex
    \usepackage{grffile} % extends the file name processing of package graphics 
                         % to support a larger range
    \makeatletter % fix for grffile with XeLaTeX
    \def\Gread@@xetex#1{%
      \IfFileExists{"\Gin@base".bb}%
      {\Gread@eps{\Gin@base.bb}}%
      {\Gread@@xetex@aux#1}%
    }
    \makeatother

    % The hyperref package gives us a pdf with properly built
    % internal navigation ('pdf bookmarks' for the table of contents,
    % internal cross-reference links, web links for URLs, etc.)
    \usepackage{hyperref}
    % The default LaTeX title has an obnoxious amount of whitespace. By default,
    % titling removes some of it. It also provides customization options.
    \usepackage{titling}
    \usepackage{longtable} % longtable support required by pandoc >1.10
    \usepackage{booktabs}  % table support for pandoc > 1.12.2
    \usepackage[inline]{enumitem} % IRkernel/repr support (it uses the enumerate* environment)
    \usepackage[normalem]{ulem} % ulem is needed to support strikethroughs (\sout)
                                % normalem makes italics be italics, not underlines
    \usepackage{mathrsfs}
    

    
    % Colors for the hyperref package
    \definecolor{urlcolor}{rgb}{0,.145,.698}
    \definecolor{linkcolor}{rgb}{.71,0.21,0.01}
    \definecolor{citecolor}{rgb}{.12,.54,.11}

    % ANSI colors
    \definecolor{ansi-black}{HTML}{3E424D}
    \definecolor{ansi-black-intense}{HTML}{282C36}
    \definecolor{ansi-red}{HTML}{E75C58}
    \definecolor{ansi-red-intense}{HTML}{B22B31}
    \definecolor{ansi-green}{HTML}{00A250}
    \definecolor{ansi-green-intense}{HTML}{007427}
    \definecolor{ansi-yellow}{HTML}{DDB62B}
    \definecolor{ansi-yellow-intense}{HTML}{B27D12}
    \definecolor{ansi-blue}{HTML}{208FFB}
    \definecolor{ansi-blue-intense}{HTML}{0065CA}
    \definecolor{ansi-magenta}{HTML}{D160C4}
    \definecolor{ansi-magenta-intense}{HTML}{A03196}
    \definecolor{ansi-cyan}{HTML}{60C6C8}
    \definecolor{ansi-cyan-intense}{HTML}{258F8F}
    \definecolor{ansi-white}{HTML}{C5C1B4}
    \definecolor{ansi-white-intense}{HTML}{A1A6B2}
    \definecolor{ansi-default-inverse-fg}{HTML}{FFFFFF}
    \definecolor{ansi-default-inverse-bg}{HTML}{000000}

    % commands and environments needed by pandoc snippets
    % extracted from the output of `pandoc -s`
    \providecommand{\tightlist}{%
      \setlength{\itemsep}{0pt}\setlength{\parskip}{0pt}}
    \DefineVerbatimEnvironment{Highlighting}{Verbatim}{commandchars=\\\{\}}
    % Add ',fontsize=\small' for more characters per line
    \newenvironment{Shaded}{}{}
    \newcommand{\KeywordTok}[1]{\textcolor[rgb]{0.00,0.44,0.13}{\textbf{{#1}}}}
    \newcommand{\DataTypeTok}[1]{\textcolor[rgb]{0.56,0.13,0.00}{{#1}}}
    \newcommand{\DecValTok}[1]{\textcolor[rgb]{0.25,0.63,0.44}{{#1}}}
    \newcommand{\BaseNTok}[1]{\textcolor[rgb]{0.25,0.63,0.44}{{#1}}}
    \newcommand{\FloatTok}[1]{\textcolor[rgb]{0.25,0.63,0.44}{{#1}}}
    \newcommand{\CharTok}[1]{\textcolor[rgb]{0.25,0.44,0.63}{{#1}}}
    \newcommand{\StringTok}[1]{\textcolor[rgb]{0.25,0.44,0.63}{{#1}}}
    \newcommand{\CommentTok}[1]{\textcolor[rgb]{0.38,0.63,0.69}{\textit{{#1}}}}
    \newcommand{\OtherTok}[1]{\textcolor[rgb]{0.00,0.44,0.13}{{#1}}}
    \newcommand{\AlertTok}[1]{\textcolor[rgb]{1.00,0.00,0.00}{\textbf{{#1}}}}
    \newcommand{\FunctionTok}[1]{\textcolor[rgb]{0.02,0.16,0.49}{{#1}}}
    \newcommand{\RegionMarkerTok}[1]{{#1}}
    \newcommand{\ErrorTok}[1]{\textcolor[rgb]{1.00,0.00,0.00}{\textbf{{#1}}}}
    \newcommand{\NormalTok}[1]{{#1}}
    
    % Additional commands for more recent versions of Pandoc
    \newcommand{\ConstantTok}[1]{\textcolor[rgb]{0.53,0.00,0.00}{{#1}}}
    \newcommand{\SpecialCharTok}[1]{\textcolor[rgb]{0.25,0.44,0.63}{{#1}}}
    \newcommand{\VerbatimStringTok}[1]{\textcolor[rgb]{0.25,0.44,0.63}{{#1}}}
    \newcommand{\SpecialStringTok}[1]{\textcolor[rgb]{0.73,0.40,0.53}{{#1}}}
    \newcommand{\ImportTok}[1]{{#1}}
    \newcommand{\DocumentationTok}[1]{\textcolor[rgb]{0.73,0.13,0.13}{\textit{{#1}}}}
    \newcommand{\AnnotationTok}[1]{\textcolor[rgb]{0.38,0.63,0.69}{\textbf{\textit{{#1}}}}}
    \newcommand{\CommentVarTok}[1]{\textcolor[rgb]{0.38,0.63,0.69}{\textbf{\textit{{#1}}}}}
    \newcommand{\VariableTok}[1]{\textcolor[rgb]{0.10,0.09,0.49}{{#1}}}
    \newcommand{\ControlFlowTok}[1]{\textcolor[rgb]{0.00,0.44,0.13}{\textbf{{#1}}}}
    \newcommand{\OperatorTok}[1]{\textcolor[rgb]{0.40,0.40,0.40}{{#1}}}
    \newcommand{\BuiltInTok}[1]{{#1}}
    \newcommand{\ExtensionTok}[1]{{#1}}
    \newcommand{\PreprocessorTok}[1]{\textcolor[rgb]{0.74,0.48,0.00}{{#1}}}
    \newcommand{\AttributeTok}[1]{\textcolor[rgb]{0.49,0.56,0.16}{{#1}}}
    \newcommand{\InformationTok}[1]{\textcolor[rgb]{0.38,0.63,0.69}{\textbf{\textit{{#1}}}}}
    \newcommand{\WarningTok}[1]{\textcolor[rgb]{0.38,0.63,0.69}{\textbf{\textit{{#1}}}}}
    
    
    % Define a nice break command that doesn't care if a line doesn't already
    % exist.
    \def\br{\hspace*{\fill} \\* }
    % Math Jax compatibility definitions
    \def\gt{>}
    \def\lt{<}
    \let\Oldtex\TeX
    \let\Oldlatex\LaTeX
    \renewcommand{\TeX}{\textrm{\Oldtex}}
    \renewcommand{\LaTeX}{\textrm{\Oldlatex}}
    % Document parameters
    % Document title
    \title{041\_clustering}
    
    
    
    
    
% Pygments definitions
\makeatletter
\def\PY@reset{\let\PY@it=\relax \let\PY@bf=\relax%
    \let\PY@ul=\relax \let\PY@tc=\relax%
    \let\PY@bc=\relax \let\PY@ff=\relax}
\def\PY@tok#1{\csname PY@tok@#1\endcsname}
\def\PY@toks#1+{\ifx\relax#1\empty\else%
    \PY@tok{#1}\expandafter\PY@toks\fi}
\def\PY@do#1{\PY@bc{\PY@tc{\PY@ul{%
    \PY@it{\PY@bf{\PY@ff{#1}}}}}}}
\def\PY#1#2{\PY@reset\PY@toks#1+\relax+\PY@do{#2}}

\expandafter\def\csname PY@tok@w\endcsname{\def\PY@tc##1{\textcolor[rgb]{0.73,0.73,0.73}{##1}}}
\expandafter\def\csname PY@tok@c\endcsname{\let\PY@it=\textit\def\PY@tc##1{\textcolor[rgb]{0.25,0.50,0.50}{##1}}}
\expandafter\def\csname PY@tok@cp\endcsname{\def\PY@tc##1{\textcolor[rgb]{0.74,0.48,0.00}{##1}}}
\expandafter\def\csname PY@tok@k\endcsname{\let\PY@bf=\textbf\def\PY@tc##1{\textcolor[rgb]{0.00,0.50,0.00}{##1}}}
\expandafter\def\csname PY@tok@kp\endcsname{\def\PY@tc##1{\textcolor[rgb]{0.00,0.50,0.00}{##1}}}
\expandafter\def\csname PY@tok@kt\endcsname{\def\PY@tc##1{\textcolor[rgb]{0.69,0.00,0.25}{##1}}}
\expandafter\def\csname PY@tok@o\endcsname{\def\PY@tc##1{\textcolor[rgb]{0.40,0.40,0.40}{##1}}}
\expandafter\def\csname PY@tok@ow\endcsname{\let\PY@bf=\textbf\def\PY@tc##1{\textcolor[rgb]{0.67,0.13,1.00}{##1}}}
\expandafter\def\csname PY@tok@nb\endcsname{\def\PY@tc##1{\textcolor[rgb]{0.00,0.50,0.00}{##1}}}
\expandafter\def\csname PY@tok@nf\endcsname{\def\PY@tc##1{\textcolor[rgb]{0.00,0.00,1.00}{##1}}}
\expandafter\def\csname PY@tok@nc\endcsname{\let\PY@bf=\textbf\def\PY@tc##1{\textcolor[rgb]{0.00,0.00,1.00}{##1}}}
\expandafter\def\csname PY@tok@nn\endcsname{\let\PY@bf=\textbf\def\PY@tc##1{\textcolor[rgb]{0.00,0.00,1.00}{##1}}}
\expandafter\def\csname PY@tok@ne\endcsname{\let\PY@bf=\textbf\def\PY@tc##1{\textcolor[rgb]{0.82,0.25,0.23}{##1}}}
\expandafter\def\csname PY@tok@nv\endcsname{\def\PY@tc##1{\textcolor[rgb]{0.10,0.09,0.49}{##1}}}
\expandafter\def\csname PY@tok@no\endcsname{\def\PY@tc##1{\textcolor[rgb]{0.53,0.00,0.00}{##1}}}
\expandafter\def\csname PY@tok@nl\endcsname{\def\PY@tc##1{\textcolor[rgb]{0.63,0.63,0.00}{##1}}}
\expandafter\def\csname PY@tok@ni\endcsname{\let\PY@bf=\textbf\def\PY@tc##1{\textcolor[rgb]{0.60,0.60,0.60}{##1}}}
\expandafter\def\csname PY@tok@na\endcsname{\def\PY@tc##1{\textcolor[rgb]{0.49,0.56,0.16}{##1}}}
\expandafter\def\csname PY@tok@nt\endcsname{\let\PY@bf=\textbf\def\PY@tc##1{\textcolor[rgb]{0.00,0.50,0.00}{##1}}}
\expandafter\def\csname PY@tok@nd\endcsname{\def\PY@tc##1{\textcolor[rgb]{0.67,0.13,1.00}{##1}}}
\expandafter\def\csname PY@tok@s\endcsname{\def\PY@tc##1{\textcolor[rgb]{0.73,0.13,0.13}{##1}}}
\expandafter\def\csname PY@tok@sd\endcsname{\let\PY@it=\textit\def\PY@tc##1{\textcolor[rgb]{0.73,0.13,0.13}{##1}}}
\expandafter\def\csname PY@tok@si\endcsname{\let\PY@bf=\textbf\def\PY@tc##1{\textcolor[rgb]{0.73,0.40,0.53}{##1}}}
\expandafter\def\csname PY@tok@se\endcsname{\let\PY@bf=\textbf\def\PY@tc##1{\textcolor[rgb]{0.73,0.40,0.13}{##1}}}
\expandafter\def\csname PY@tok@sr\endcsname{\def\PY@tc##1{\textcolor[rgb]{0.73,0.40,0.53}{##1}}}
\expandafter\def\csname PY@tok@ss\endcsname{\def\PY@tc##1{\textcolor[rgb]{0.10,0.09,0.49}{##1}}}
\expandafter\def\csname PY@tok@sx\endcsname{\def\PY@tc##1{\textcolor[rgb]{0.00,0.50,0.00}{##1}}}
\expandafter\def\csname PY@tok@m\endcsname{\def\PY@tc##1{\textcolor[rgb]{0.40,0.40,0.40}{##1}}}
\expandafter\def\csname PY@tok@gh\endcsname{\let\PY@bf=\textbf\def\PY@tc##1{\textcolor[rgb]{0.00,0.00,0.50}{##1}}}
\expandafter\def\csname PY@tok@gu\endcsname{\let\PY@bf=\textbf\def\PY@tc##1{\textcolor[rgb]{0.50,0.00,0.50}{##1}}}
\expandafter\def\csname PY@tok@gd\endcsname{\def\PY@tc##1{\textcolor[rgb]{0.63,0.00,0.00}{##1}}}
\expandafter\def\csname PY@tok@gi\endcsname{\def\PY@tc##1{\textcolor[rgb]{0.00,0.63,0.00}{##1}}}
\expandafter\def\csname PY@tok@gr\endcsname{\def\PY@tc##1{\textcolor[rgb]{1.00,0.00,0.00}{##1}}}
\expandafter\def\csname PY@tok@ge\endcsname{\let\PY@it=\textit}
\expandafter\def\csname PY@tok@gs\endcsname{\let\PY@bf=\textbf}
\expandafter\def\csname PY@tok@gp\endcsname{\let\PY@bf=\textbf\def\PY@tc##1{\textcolor[rgb]{0.00,0.00,0.50}{##1}}}
\expandafter\def\csname PY@tok@go\endcsname{\def\PY@tc##1{\textcolor[rgb]{0.53,0.53,0.53}{##1}}}
\expandafter\def\csname PY@tok@gt\endcsname{\def\PY@tc##1{\textcolor[rgb]{0.00,0.27,0.87}{##1}}}
\expandafter\def\csname PY@tok@err\endcsname{\def\PY@bc##1{\setlength{\fboxsep}{0pt}\fcolorbox[rgb]{1.00,0.00,0.00}{1,1,1}{\strut ##1}}}
\expandafter\def\csname PY@tok@kc\endcsname{\let\PY@bf=\textbf\def\PY@tc##1{\textcolor[rgb]{0.00,0.50,0.00}{##1}}}
\expandafter\def\csname PY@tok@kd\endcsname{\let\PY@bf=\textbf\def\PY@tc##1{\textcolor[rgb]{0.00,0.50,0.00}{##1}}}
\expandafter\def\csname PY@tok@kn\endcsname{\let\PY@bf=\textbf\def\PY@tc##1{\textcolor[rgb]{0.00,0.50,0.00}{##1}}}
\expandafter\def\csname PY@tok@kr\endcsname{\let\PY@bf=\textbf\def\PY@tc##1{\textcolor[rgb]{0.00,0.50,0.00}{##1}}}
\expandafter\def\csname PY@tok@bp\endcsname{\def\PY@tc##1{\textcolor[rgb]{0.00,0.50,0.00}{##1}}}
\expandafter\def\csname PY@tok@fm\endcsname{\def\PY@tc##1{\textcolor[rgb]{0.00,0.00,1.00}{##1}}}
\expandafter\def\csname PY@tok@vc\endcsname{\def\PY@tc##1{\textcolor[rgb]{0.10,0.09,0.49}{##1}}}
\expandafter\def\csname PY@tok@vg\endcsname{\def\PY@tc##1{\textcolor[rgb]{0.10,0.09,0.49}{##1}}}
\expandafter\def\csname PY@tok@vi\endcsname{\def\PY@tc##1{\textcolor[rgb]{0.10,0.09,0.49}{##1}}}
\expandafter\def\csname PY@tok@vm\endcsname{\def\PY@tc##1{\textcolor[rgb]{0.10,0.09,0.49}{##1}}}
\expandafter\def\csname PY@tok@sa\endcsname{\def\PY@tc##1{\textcolor[rgb]{0.73,0.13,0.13}{##1}}}
\expandafter\def\csname PY@tok@sb\endcsname{\def\PY@tc##1{\textcolor[rgb]{0.73,0.13,0.13}{##1}}}
\expandafter\def\csname PY@tok@sc\endcsname{\def\PY@tc##1{\textcolor[rgb]{0.73,0.13,0.13}{##1}}}
\expandafter\def\csname PY@tok@dl\endcsname{\def\PY@tc##1{\textcolor[rgb]{0.73,0.13,0.13}{##1}}}
\expandafter\def\csname PY@tok@s2\endcsname{\def\PY@tc##1{\textcolor[rgb]{0.73,0.13,0.13}{##1}}}
\expandafter\def\csname PY@tok@sh\endcsname{\def\PY@tc##1{\textcolor[rgb]{0.73,0.13,0.13}{##1}}}
\expandafter\def\csname PY@tok@s1\endcsname{\def\PY@tc##1{\textcolor[rgb]{0.73,0.13,0.13}{##1}}}
\expandafter\def\csname PY@tok@mb\endcsname{\def\PY@tc##1{\textcolor[rgb]{0.40,0.40,0.40}{##1}}}
\expandafter\def\csname PY@tok@mf\endcsname{\def\PY@tc##1{\textcolor[rgb]{0.40,0.40,0.40}{##1}}}
\expandafter\def\csname PY@tok@mh\endcsname{\def\PY@tc##1{\textcolor[rgb]{0.40,0.40,0.40}{##1}}}
\expandafter\def\csname PY@tok@mi\endcsname{\def\PY@tc##1{\textcolor[rgb]{0.40,0.40,0.40}{##1}}}
\expandafter\def\csname PY@tok@il\endcsname{\def\PY@tc##1{\textcolor[rgb]{0.40,0.40,0.40}{##1}}}
\expandafter\def\csname PY@tok@mo\endcsname{\def\PY@tc##1{\textcolor[rgb]{0.40,0.40,0.40}{##1}}}
\expandafter\def\csname PY@tok@ch\endcsname{\let\PY@it=\textit\def\PY@tc##1{\textcolor[rgb]{0.25,0.50,0.50}{##1}}}
\expandafter\def\csname PY@tok@cm\endcsname{\let\PY@it=\textit\def\PY@tc##1{\textcolor[rgb]{0.25,0.50,0.50}{##1}}}
\expandafter\def\csname PY@tok@cpf\endcsname{\let\PY@it=\textit\def\PY@tc##1{\textcolor[rgb]{0.25,0.50,0.50}{##1}}}
\expandafter\def\csname PY@tok@c1\endcsname{\let\PY@it=\textit\def\PY@tc##1{\textcolor[rgb]{0.25,0.50,0.50}{##1}}}
\expandafter\def\csname PY@tok@cs\endcsname{\let\PY@it=\textit\def\PY@tc##1{\textcolor[rgb]{0.25,0.50,0.50}{##1}}}

\def\PYZbs{\char`\\}
\def\PYZus{\char`\_}
\def\PYZob{\char`\{}
\def\PYZcb{\char`\}}
\def\PYZca{\char`\^}
\def\PYZam{\char`\&}
\def\PYZlt{\char`\<}
\def\PYZgt{\char`\>}
\def\PYZsh{\char`\#}
\def\PYZpc{\char`\%}
\def\PYZdl{\char`\$}
\def\PYZhy{\char`\-}
\def\PYZsq{\char`\'}
\def\PYZdq{\char`\"}
\def\PYZti{\char`\~}
% for compatibility with earlier versions
\def\PYZat{@}
\def\PYZlb{[}
\def\PYZrb{]}
\makeatother


    % For linebreaks inside Verbatim environment from package fancyvrb. 
    \makeatletter
        \newbox\Wrappedcontinuationbox 
        \newbox\Wrappedvisiblespacebox 
        \newcommand*\Wrappedvisiblespace {\textcolor{red}{\textvisiblespace}} 
        \newcommand*\Wrappedcontinuationsymbol {\textcolor{red}{\llap{\tiny$\m@th\hookrightarrow$}}} 
        \newcommand*\Wrappedcontinuationindent {3ex } 
        \newcommand*\Wrappedafterbreak {\kern\Wrappedcontinuationindent\copy\Wrappedcontinuationbox} 
        % Take advantage of the already applied Pygments mark-up to insert 
        % potential linebreaks for TeX processing. 
        %        {, <, #, %, $, ' and ": go to next line. 
        %        _, }, ^, &, >, - and ~: stay at end of broken line. 
        % Use of \textquotesingle for straight quote. 
        \newcommand*\Wrappedbreaksatspecials {% 
            \def\PYGZus{\discretionary{\char`\_}{\Wrappedafterbreak}{\char`\_}}% 
            \def\PYGZob{\discretionary{}{\Wrappedafterbreak\char`\{}{\char`\{}}% 
            \def\PYGZcb{\discretionary{\char`\}}{\Wrappedafterbreak}{\char`\}}}% 
            \def\PYGZca{\discretionary{\char`\^}{\Wrappedafterbreak}{\char`\^}}% 
            \def\PYGZam{\discretionary{\char`\&}{\Wrappedafterbreak}{\char`\&}}% 
            \def\PYGZlt{\discretionary{}{\Wrappedafterbreak\char`\<}{\char`\<}}% 
            \def\PYGZgt{\discretionary{\char`\>}{\Wrappedafterbreak}{\char`\>}}% 
            \def\PYGZsh{\discretionary{}{\Wrappedafterbreak\char`\#}{\char`\#}}% 
            \def\PYGZpc{\discretionary{}{\Wrappedafterbreak\char`\%}{\char`\%}}% 
            \def\PYGZdl{\discretionary{}{\Wrappedafterbreak\char`\$}{\char`\$}}% 
            \def\PYGZhy{\discretionary{\char`\-}{\Wrappedafterbreak}{\char`\-}}% 
            \def\PYGZsq{\discretionary{}{\Wrappedafterbreak\textquotesingle}{\textquotesingle}}% 
            \def\PYGZdq{\discretionary{}{\Wrappedafterbreak\char`\"}{\char`\"}}% 
            \def\PYGZti{\discretionary{\char`\~}{\Wrappedafterbreak}{\char`\~}}% 
        } 
        % Some characters . , ; ? ! / are not pygmentized. 
        % This macro makes them "active" and they will insert potential linebreaks 
        \newcommand*\Wrappedbreaksatpunct {% 
            \lccode`\~`\.\lowercase{\def~}{\discretionary{\hbox{\char`\.}}{\Wrappedafterbreak}{\hbox{\char`\.}}}% 
            \lccode`\~`\,\lowercase{\def~}{\discretionary{\hbox{\char`\,}}{\Wrappedafterbreak}{\hbox{\char`\,}}}% 
            \lccode`\~`\;\lowercase{\def~}{\discretionary{\hbox{\char`\;}}{\Wrappedafterbreak}{\hbox{\char`\;}}}% 
            \lccode`\~`\:\lowercase{\def~}{\discretionary{\hbox{\char`\:}}{\Wrappedafterbreak}{\hbox{\char`\:}}}% 
            \lccode`\~`\?\lowercase{\def~}{\discretionary{\hbox{\char`\?}}{\Wrappedafterbreak}{\hbox{\char`\?}}}% 
            \lccode`\~`\!\lowercase{\def~}{\discretionary{\hbox{\char`\!}}{\Wrappedafterbreak}{\hbox{\char`\!}}}% 
            \lccode`\~`\/\lowercase{\def~}{\discretionary{\hbox{\char`\/}}{\Wrappedafterbreak}{\hbox{\char`\/}}}% 
            \catcode`\.\active
            \catcode`\,\active 
            \catcode`\;\active
            \catcode`\:\active
            \catcode`\?\active
            \catcode`\!\active
            \catcode`\/\active 
            \lccode`\~`\~ 	
        }
    \makeatother

    \let\OriginalVerbatim=\Verbatim
    \makeatletter
    \renewcommand{\Verbatim}[1][1]{%
        %\parskip\z@skip
        \sbox\Wrappedcontinuationbox {\Wrappedcontinuationsymbol}%
        \sbox\Wrappedvisiblespacebox {\FV@SetupFont\Wrappedvisiblespace}%
        \def\FancyVerbFormatLine ##1{\hsize\linewidth
            \vtop{\raggedright\hyphenpenalty\z@\exhyphenpenalty\z@
                \doublehyphendemerits\z@\finalhyphendemerits\z@
                \strut ##1\strut}%
        }%
        % If the linebreak is at a space, the latter will be displayed as visible
        % space at end of first line, and a continuation symbol starts next line.
        % Stretch/shrink are however usually zero for typewriter font.
        \def\FV@Space {%
            \nobreak\hskip\z@ plus\fontdimen3\font minus\fontdimen4\font
            \discretionary{\copy\Wrappedvisiblespacebox}{\Wrappedafterbreak}
            {\kern\fontdimen2\font}%
        }%
        
        % Allow breaks at special characters using \PYG... macros.
        \Wrappedbreaksatspecials
        % Breaks at punctuation characters . , ; ? ! and / need catcode=\active 	
        \OriginalVerbatim[#1,codes*=\Wrappedbreaksatpunct]%
    }
    \makeatother

    % Exact colors from NB
    \definecolor{incolor}{HTML}{303F9F}
    \definecolor{outcolor}{HTML}{D84315}
    \definecolor{cellborder}{HTML}{CFCFCF}
    \definecolor{cellbackground}{HTML}{F7F7F7}
    
    % prompt
    \makeatletter
    \newcommand{\boxspacing}{\kern\kvtcb@left@rule\kern\kvtcb@boxsep}
    \makeatother
    \newcommand{\prompt}[4]{
        \ttfamily\llap{{\color{#2}[#3]:\hspace{3pt}#4}}\vspace{-\baselineskip}
    }
    

    
    % Prevent overflowing lines due to hard-to-break entities
    \sloppy 
    % Setup hyperref package
    \hypersetup{
      breaklinks=true,  % so long urls are correctly broken across lines
      colorlinks=true,
      urlcolor=urlcolor,
      linkcolor=linkcolor,
      citecolor=citecolor,
      }
    % Slightly bigger margins than the latex defaults
    
    \geometry{verbose,tmargin=1in,bmargin=1in,lmargin=1in,rmargin=1in}
    
    

\begin{document}
    
    \maketitle
    
    

    
    \section{Title: Graph Clustering and Similarity
Analysis}\label{title-graph-clustering-and-similarity-analysis}

\subsubsection{Purpose: Can we apply graph clustering and similarity
methods to group models by valid cyber security
properties?}\label{purpose-can-we-apply-graph-clustering-and-similarity-methods-to-group-models-by-valid-cyber-security-properties}

\subsubsection{Author: @mjz}\label{author-mjz}

    \subsection{Global imports and notebook setup
here}\label{global-imports-and-notebook-setup-here}

    \begin{tcolorbox}[breakable, size=fbox, boxrule=1pt, pad at break*=1mm,colback=cellbackground, colframe=cellborder]
\prompt{In}{incolor}{6}{\boxspacing}
\begin{Verbatim}[commandchars=\\\{\}]
\PY{o}{\PYZpc{}}\PY{k}{matplotlib} inline

\PY{c+c1}{\PYZsh{}\PYZhy{}\PYZhy{}\PYZhy{}\PYZhy{}\PYZhy{}\PYZhy{} python libs \PYZhy{}\PYZhy{}\PYZhy{}\PYZhy{}\PYZhy{}\PYZhy{}\PYZhy{}\PYZhy{}\PYZhy{}\PYZhy{}}
\PY{k+kn}{import} \PY{n+nn}{logging}
\PY{k+kn}{import} \PY{n+nn}{os}
\PY{k+kn}{import} \PY{n+nn}{sys}
\PY{k+kn}{import} \PY{n+nn}{uuid}
\PY{k+kn}{import} \PY{n+nn}{pathlib}

\PY{c+c1}{\PYZsh{}\PYZhy{}\PYZhy{}\PYZhy{}\PYZhy{}\PYZhy{}\PYZhy{} notebook libs \PYZhy{}\PYZhy{}\PYZhy{}\PYZhy{}\PYZhy{}\PYZhy{}\PYZhy{}\PYZhy{}\PYZhy{}\PYZhy{}}
\PY{k+kn}{import} \PY{n+nn}{matplotlib}\PY{n+nn}{.}\PY{n+nn}{pyplot} \PY{k}{as} \PY{n+nn}{plt}
\PY{k+kn}{import} \PY{n+nn}{networkx} \PY{k}{as} \PY{n+nn}{nx}
\PY{k+kn}{import} \PY{n+nn}{graphviz}
\PY{k+kn}{import} \PY{n+nn}{IPython}\PY{n+nn}{.}\PY{n+nn}{display}
\PY{k+kn}{import} \PY{n+nn}{numpy} \PY{k}{as} \PY{n+nn}{np}
\PY{c+c1}{\PYZsh{} import scipy.stats as st}
\PY{c+c1}{\PYZsh{} import scipy.special}
\PY{c+c1}{\PYZsh{} \PYZsh{} bokeh plotting setup}
\PY{c+c1}{\PYZsh{} import bokeh.io}
\PY{c+c1}{\PYZsh{} import bokeh.plotting}
\PY{c+c1}{\PYZsh{} import bokeh.application}
\PY{c+c1}{\PYZsh{} import bokeh.application.handlers}
\PY{c+c1}{\PYZsh{} bokeh.io.output\PYZus{}notebook()}
\PY{c+c1}{\PYZsh{} notebook\PYZus{}url = \PYZsq{}localhost:8888\PYZsq{}}


\PY{c+c1}{\PYZsh{}\PYZhy{}\PYZhy{}\PYZhy{}\PYZhy{}\PYZhy{}\PYZhy{} project libs \PYZhy{}\PYZhy{}\PYZhy{}\PYZhy{}\PYZhy{}\PYZhy{}\PYZhy{}\PYZhy{}\PYZhy{}\PYZhy{}}
\PY{n}{py\PYZus{}mulval\PYZus{}path} \PY{o}{=} \PY{l+s+sa}{r}\PY{l+s+s1}{\PYZsq{}}\PY{l+s+s1}{/opt/projects/diss/py\PYZhy{}mulval/src}\PY{l+s+s1}{\PYZsq{}}
\PY{n}{sys}\PY{o}{.}\PY{n}{path}\PY{o}{.}\PY{n}{append}\PY{p}{(}\PY{n}{py\PYZus{}mulval\PYZus{}path}\PY{p}{)}

\PY{c+c1}{\PYZsh{} Just importing for their flags right now}
\PY{k+kn}{import} \PY{n+nn}{py\PYZus{}mulval}\PY{n+nn}{.}\PY{n+nn}{py\PYZus{}mulval} \PY{k}{as} \PY{n+nn}{py\PYZus{}mulval}
\PY{k+kn}{import} \PY{n+nn}{py\PYZus{}mulval}\PY{n+nn}{.}\PY{n+nn}{boromir} \PY{k}{as} \PY{n+nn}{boromir}
\PY{k+kn}{import} \PY{n+nn}{py\PYZus{}mulval}\PY{n+nn}{.}\PY{n+nn}{log\PYZus{}util} \PY{k}{as} \PY{n+nn}{log\PYZus{}util}
\PY{k+kn}{import} \PY{n+nn}{py\PYZus{}mulval}\PY{n+nn}{.}\PY{n+nn}{mulpy} \PY{k}{as} \PY{n+nn}{mulpy}
\PY{k+kn}{import} \PY{n+nn}{py\PYZus{}mulval}\PY{n+nn}{.}\PY{n+nn}{metrics} \PY{k}{as} \PY{n+nn}{secmet}
\PY{k+kn}{import} \PY{n+nn}{py\PYZus{}mulval}\PY{n+nn}{.}\PY{n+nn}{attack\PYZus{}graph} \PY{k}{as} \PY{n+nn}{attack\PYZus{}graph}
\PY{k+kn}{import} \PY{n+nn}{py\PYZus{}mulval}\PY{n+nn}{.}\PY{n+nn}{mulval\PYZus{}fact\PYZus{}graph} \PY{k}{as} \PY{n+nn}{fact\PYZus{}graph} 

\PY{k+kn}{from} \PY{n+nn}{py\PYZus{}mulval} \PY{k+kn}{import} \PY{n}{flags}
\PY{n}{FLAGS} \PY{o}{=} \PY{n}{flags}\PY{o}{.}\PY{n}{FLAGS}
\PY{n}{FLAGS}\PY{p}{(}\PY{p}{[}\PY{n}{sys}\PY{o}{.}\PY{n}{argv}\PY{p}{[}\PY{l+m+mi}{0}\PY{p}{]}\PY{p}{]}\PY{p}{)} \PY{c+c1}{\PYZsh{} dont expect cli args from jupyter}
\end{Verbatim}
\end{tcolorbox}

            \begin{tcolorbox}[breakable, size=fbox, boxrule=.5pt, pad at break*=1mm, opacityfill=0]
\prompt{Out}{outcolor}{6}{\boxspacing}
\begin{Verbatim}[commandchars=\\\{\}]
['/opt/projects/diss/jupyter\_nbs/py36/lib/python3.6/site-
packages/ipykernel\_launcher.py']
\end{Verbatim}
\end{tcolorbox}
        
    \begin{tcolorbox}[breakable, size=fbox, boxrule=1pt, pad at break*=1mm,colback=cellbackground, colframe=cellborder]
\prompt{In}{incolor}{13}{\boxspacing}
\begin{Verbatim}[commandchars=\\\{\}]
\PY{c+c1}{\PYZsh{} Show a nx\PYZus{}Agraph in this notebook}
\PY{k}{def} \PY{n+nf}{plot\PYZus{}gv}\PY{p}{(}\PY{n}{g}\PY{p}{,} \PY{n}{title}\PY{o}{=}\PY{k+kc}{None}\PY{p}{)}\PY{p}{:}
    \PY{n}{A} \PY{o}{=} \PY{n}{nx}\PY{o}{.}\PY{n}{nx\PYZus{}agraph}\PY{o}{.}\PY{n}{to\PYZus{}agraph}\PY{p}{(}\PY{n}{g}\PY{p}{)}
    \PY{k}{if} \PY{n}{title}\PY{p}{:}
        \PY{n}{A}\PY{o}{.}\PY{n}{graph\PYZus{}attr}\PY{o}{.}\PY{n}{update}\PY{p}{(}\PY{n}{label}\PY{o}{=}\PY{n}{title}\PY{p}{,} \PY{n}{labelloc}\PY{o}{=}\PY{l+s+s1}{\PYZsq{}}\PY{l+s+s1}{top}\PY{l+s+s1}{\PYZsq{}}\PY{p}{,} \PY{n}{labeljust}\PY{o}{=}\PY{l+s+s1}{\PYZsq{}}\PY{l+s+s1}{center}\PY{l+s+s1}{\PYZsq{}}\PY{p}{,} \PY{n}{fontsize}\PY{o}{=}\PY{l+m+mi}{24}\PY{p}{)}
    \PY{c+c1}{\PYZsh{} A.layout(\PYZsq{}dot\PYZsq{}, args=\PYZsq{}\PYZhy{}Nfontsize=10 \PYZhy{}Nwidth=\PYZdq{}.2\PYZdq{} \PYZhy{}Nheight=\PYZdq{}.2\PYZdq{} \PYZhy{}Nmargin=0 \PYZhy{}Gfontsize=8\PYZsq{})}
    \PY{n}{args} \PY{o}{=} \PY{l+s+s2}{\PYZdq{}\PYZdq{}\PYZdq{}}
\PY{l+s+s2}{    \PYZhy{}Gsize=17}
\PY{l+s+s2}{    \PYZhy{}Nfontname=Roboto}
\PY{l+s+s2}{    \PYZhy{}Efontname=Roboto}
\PY{l+s+s2}{    \PYZhy{}Nfontsize=14}
\PY{l+s+s2}{    \PYZhy{}Efontsize=14}
\PY{l+s+s2}{    }\PY{l+s+s2}{\PYZdq{}\PYZdq{}\PYZdq{}}
    \PY{n}{A}\PY{o}{.}\PY{n}{layout}\PY{p}{(}\PY{l+s+s1}{\PYZsq{}}\PY{l+s+s1}{dot}\PY{l+s+s1}{\PYZsq{}}\PY{p}{,} \PY{n}{args}\PY{o}{=}\PY{n}{args} \PY{p}{)}
    \PY{c+c1}{\PYZsh{} A.draw(self.outputDir + \PYZsq{}/\PYZsq{} + outfilename)}
    \PY{n}{A}\PY{o}{.}\PY{n}{draw}\PY{p}{(}\PY{p}{)}
    \PY{c+c1}{\PYZsh{} plt.show()}
    \PY{k}{return} \PY{n}{graphviz}\PY{o}{.}\PY{n}{Source}\PY{p}{(}\PY{n}{A}\PY{o}{.}\PY{n}{to\PYZus{}string}\PY{p}{(}\PY{p}{)}\PY{p}{)}
\end{Verbatim}
\end{tcolorbox}

    \subsection{Problem \& Background Info}\label{problem-background-info}

    \begin{tcolorbox}[breakable, size=fbox, boxrule=1pt, pad at break*=1mm,colback=cellbackground, colframe=cellborder]
\prompt{In}{incolor}{ }{\boxspacing}
\begin{Verbatim}[commandchars=\\\{\}]

\end{Verbatim}
\end{tcolorbox}

    \subsection{NetLSD - Graph Fingerprint
Distance}\label{netlsd---graph-fingerprint-distance}

https://github.com/xgfs/NetLSD

@inproceedings\{Tsitsulin:2018:KDD, author=\{Tsitsulin, Anton and
Mottin, Davide and Karras, Panagiotis and Bronstein, Alex and
M\{"u\}ller, Emmanuel\}, title=\{NetLSD: Hearing the Shape of a Graph\},
booktitle = \{Proceedings of the 24th ACM SIGKDD International
Conference on Knowledge Discovery and Data Mining\}, series = \{KDD
'18\}, year = \{2018\}, \}

    \begin{tcolorbox}[breakable, size=fbox, boxrule=1pt, pad at break*=1mm,colback=cellbackground, colframe=cellborder]
\prompt{In}{incolor}{1}{\boxspacing}
\begin{Verbatim}[commandchars=\\\{\}]
\PY{k+kn}{import} \PY{n+nn}{netlsd}
\PY{k+kn}{import} \PY{n+nn}{networkx} \PY{k}{as} \PY{n+nn}{nx}
\PY{k+kn}{import} \PY{n+nn}{numpy} \PY{k}{as} \PY{n+nn}{np}

\PY{n}{g} \PY{o}{=} \PY{n}{nx}\PY{o}{.}\PY{n}{erdos\PYZus{}renyi\PYZus{}graph}\PY{p}{(}\PY{l+m+mi}{100}\PY{p}{,} \PY{l+m+mf}{0.01}\PY{p}{)} \PY{c+c1}{\PYZsh{} create a random graph with 100 nodes}
\PY{n}{desc1} \PY{o}{=} \PY{n}{netlsd}\PY{o}{.}\PY{n}{heat}\PY{p}{(}\PY{n}{g}\PY{p}{)} \PY{c+c1}{\PYZsh{} compute the signature}
\PY{n}{g} \PY{o}{=} \PY{n}{nx}\PY{o}{.}\PY{n}{erdos\PYZus{}renyi\PYZus{}graph}\PY{p}{(}\PY{l+m+mi}{100}\PY{p}{,} \PY{l+m+mf}{0.01}\PY{p}{)} \PY{c+c1}{\PYZsh{} create a random graph with 100 nodes}
\PY{n}{desc2} \PY{o}{=} \PY{n}{netlsd}\PY{o}{.}\PY{n}{heat}\PY{p}{(}\PY{n}{g}\PY{p}{)} \PY{c+c1}{\PYZsh{} compute the signature}


\PY{n}{distance1} \PY{o}{=} \PY{n}{netlsd}\PY{o}{.}\PY{n}{compare}\PY{p}{(}\PY{n}{desc1}\PY{p}{,} \PY{n}{desc2}\PY{p}{)} \PY{c+c1}{\PYZsh{} compare the signatures using l2 distance}
\PY{n+nb}{print}\PY{p}{(}\PY{n}{distance1}\PY{p}{)}
\PY{n}{distance2} \PY{o}{=} \PY{n}{np}\PY{o}{.}\PY{n}{linalg}\PY{o}{.}\PY{n}{norm}\PY{p}{(}\PY{n}{desc1} \PY{o}{\PYZhy{}} \PY{n}{desc2}\PY{p}{)} \PY{c+c1}{\PYZsh{} equivalent}
\PY{n+nb}{print}\PY{p}{(}\PY{n}{distance2}\PY{p}{)}
\end{Verbatim}
\end{tcolorbox}

    \begin{Verbatim}[commandchars=\\\{\}]
0.754599090929082
0.754599090929082
    \end{Verbatim}

    \subsection{Test AG/FG distance}\label{test-agfg-distance}

    \begin{tcolorbox}[breakable, size=fbox, boxrule=1pt, pad at break*=1mm,colback=cellbackground, colframe=cellborder]
\prompt{In}{incolor}{19}{\boxspacing}
\begin{Verbatim}[commandchars=\\\{\}]
\PY{n}{datapath} \PY{o}{=} \PY{l+s+s1}{\PYZsq{}}\PY{l+s+s1}{/opt/projects/diss/py\PYZhy{}mulval/data}\PY{l+s+s1}{\PYZsq{}}

\PY{n}{fg} \PY{o}{=} \PY{n}{fact\PYZus{}graph}\PY{o}{.}\PY{n}{FactGraph}\PY{p}{(}\PY{p}{)}
\PY{n}{fg}\PY{o}{.}\PY{n}{load\PYZus{}json\PYZus{}file}\PY{p}{(}\PY{l+s+s1}{\PYZsq{}}\PY{l+s+s1}{/opt/projects/diss/py\PYZhy{}mulval/data/facts/mulval\PYZus{}facts.multi\PYZus{}host\PYZus{}1.json}\PY{l+s+s1}{\PYZsq{}}\PY{p}{)}
\PY{n}{plot\PYZus{}gv}\PY{p}{(}\PY{n}{fg}\PY{p}{,} \PY{l+s+s1}{\PYZsq{}}\PY{l+s+s1}{Original Network}\PY{l+s+s1}{\PYZsq{}}\PY{p}{)}
\end{Verbatim}
\end{tcolorbox}
 
            
\prompt{Out}{outcolor}{19}{}
    
    \begin{center}
    \adjustimage{max size={0.9\linewidth}{0.9\paperheight}}{output_9_0.pdf}
    \end{center}
    { \hspace*{\fill} \\}
    

    \begin{tcolorbox}[breakable, size=fbox, boxrule=1pt, pad at break*=1mm,colback=cellbackground, colframe=cellborder]
\prompt{In}{incolor}{20}{\boxspacing}
\begin{Verbatim}[commandchars=\\\{\}]
\PY{n}{ag} \PY{o}{=} \PY{n}{attack\PYZus{}graph}\PY{o}{.}\PY{n}{AttackGraph}\PY{p}{(}\PY{p}{)} 
\PY{n}{ag}\PY{o}{.}\PY{n}{load\PYZus{}dot\PYZus{}file}\PY{p}{(}\PY{l+s+s1}{\PYZsq{}}\PY{l+s+s1}{/opt/projects/diss/py\PYZhy{}mulval/data/mulval\PYZus{}ag/small\PYZus{}enterprise/AttackGraph.dot}\PY{l+s+s1}{\PYZsq{}}\PY{p}{)}
\PY{n}{ag}\PY{o}{.}\PY{n}{load\PYZus{}score\PYZus{}dict}\PY{p}{(}\PY{l+s+s1}{\PYZsq{}}\PY{l+s+s1}{/opt/projects/diss/py\PYZhy{}mulval/data/scoreDict.yml}\PY{l+s+s1}{\PYZsq{}}\PY{p}{)}
\PY{n}{plot\PYZus{}gv}\PY{p}{(}\PY{n}{ag}\PY{p}{,} \PY{l+s+s1}{\PYZsq{}}\PY{l+s+s1}{Attack Graph}\PY{l+s+s1}{\PYZsq{}}\PY{p}{)}
\end{Verbatim}
\end{tcolorbox}
 
            
\prompt{Out}{outcolor}{20}{}
    
    \begin{center}
    \adjustimage{max size={0.9\linewidth}{0.9\paperheight}}{output_10_0.pdf}
    \end{center}
    { \hspace*{\fill} \\}
    

    \begin{tcolorbox}[breakable, size=fbox, boxrule=1pt, pad at break*=1mm,colback=cellbackground, colframe=cellborder]
\prompt{In}{incolor}{21}{\boxspacing}
\begin{Verbatim}[commandchars=\\\{\}]
\PY{n}{desc\PYZus{}ag} \PY{o}{=} \PY{n}{netlsd}\PY{o}{.}\PY{n}{heat}\PY{p}{(}\PY{n}{ag}\PY{p}{)} \PY{c+c1}{\PYZsh{} compute the signature}
\PY{n}{desc\PYZus{}fg} \PY{o}{=} \PY{n}{netlsd}\PY{o}{.}\PY{n}{heat}\PY{p}{(}\PY{n}{fg}\PY{p}{)} \PY{c+c1}{\PYZsh{} compute the signature}


\PY{n}{distance1} \PY{o}{=} \PY{n}{netlsd}\PY{o}{.}\PY{n}{compare}\PY{p}{(}\PY{n}{desc1}\PY{p}{,} \PY{n}{desc2}\PY{p}{)} \PY{c+c1}{\PYZsh{} compare the signatures using l2 distance}
\PY{n+nb}{print}\PY{p}{(}\PY{n}{distance1}\PY{p}{)}
\PY{n}{distance2} \PY{o}{=} \PY{n}{np}\PY{o}{.}\PY{n}{linalg}\PY{o}{.}\PY{n}{norm}\PY{p}{(}\PY{n}{desc1} \PY{o}{\PYZhy{}} \PY{n}{desc2}\PY{p}{)} \PY{c+c1}{\PYZsh{} equivalent}
\PY{n+nb}{print}\PY{p}{(}\PY{n}{distance2}\PY{p}{)}
\end{Verbatim}
\end{tcolorbox}

    \begin{Verbatim}[commandchars=\\\{\}]

        ---------------------------------------------------------------------------

        NetworkXNotImplemented                    Traceback (most recent call last)

        <ipython-input-21-ef79ae8b0aa2> in <module>
    ----> 1 desc\_ag = netlsd.heat(ag) \# compute the signature
          2 desc\_fg = netlsd.heat(fg) \# compute the signature
          3 
          4 
          5 distance1 = netlsd.compare(desc1, desc2) \# compare the signatures using l2 distance


        /opt/projects/diss/jupyter\_nbs/py36/lib/python3.6/site-packages/netlsd/kernels.py in heat(inp, timescales, eigenvalues, normalization, normalized\_laplacian)
        125 
        126     """
    --> 127     return netlsd(inp, timescales, 'heat', eigenvalues, normalization, normalized\_laplacian)
        128 
        129 


        /opt/projects/diss/jupyter\_nbs/py36/lib/python3.6/site-packages/netlsd/kernels.py in netlsd(inp, timescales, kernel, eigenvalues, normalization, normalized\_laplacian)
         80         mat = check\_2d(inp)
         81         if mat is None:
    ---> 82             mat = graph\_to\_laplacian(inp, normalized\_laplacian)
         83             if mat is None:
         84                 raise ValueError('Unirecognized input type: expected one of [\textbackslash{}'np.ndarray\textbackslash{}', \textbackslash{}'scipy.sparse\textbackslash{}', \textbackslash{}'networkx.Graph\textbackslash{}',\textbackslash{}' graph\_tool.Graph,\textbackslash{}' or \textbackslash{}'igraph.Graph\textbackslash{}'], got \{0\}'.format(type(inp)))


        /opt/projects/diss/jupyter\_nbs/py36/lib/python3.6/site-packages/netlsd/util.py in graph\_to\_laplacian(G, normalized)
        102         if isinstance(G, nx.Graph):
        103             if normalized:
    --> 104                 return nx.normalized\_laplacian\_matrix(G)
        105             else:
        106                 return nx.laplacian\_matrix(G)


        </opt/projects/diss/jupyter\_nbs/py36/lib/python3.6/site-packages/decorator.py:decorator-gen-795> in normalized\_laplacian\_matrix(G, nodelist, weight)


        /opt/projects/diss/jupyter\_nbs/py36/lib/python3.6/site-packages/networkx/utils/decorators.py in \_not\_implemented\_for(not\_implement\_for\_func, *args, **kwargs)
         78         if match:
         79             msg = 'not implemented for \%s type' \% ' '.join(graph\_types)
    ---> 80             raise nx.NetworkXNotImplemented(msg)
         81         else:
         82             return not\_implement\_for\_func(*args, **kwargs)


        NetworkXNotImplemented: not implemented for directed type

    \end{Verbatim}

    \subsection{NetLSD doesn't implement DAGs
Yet}\label{netlsd-doesnt-implement-dags-yet}

No problem.

    \begin{tcolorbox}[breakable, size=fbox, boxrule=1pt, pad at break*=1mm,colback=cellbackground, colframe=cellborder]
\prompt{In}{incolor}{22}{\boxspacing}
\begin{Verbatim}[commandchars=\\\{\}]
\PY{o}{\PYZpc{}\PYZpc{}}\PY{k}{bigquery} df

\PYZsh{}standardSQL
SELECT thedate, test, value, unit, metric,
\PYZhy{}\PYZhy{} citation, cite\PYZus{}key, metric, metric\PYZus{}name, metric\PYZus{}summary, metric\PYZus{}usage, 
attack\PYZus{}graph, transition\PYZus{}matrix, transition\PYZus{}matrix\PYZus{}raw,
labels
FROM (
  SELECT
\PYZsh{}     value,
    TIMESTAMP\PYZus{}MICROS(CAST(timestamp * 1000000 AS int64)) AS thedate,
    test, value, unit, labels,metric, 
    REGEXP\PYZus{}EXTRACT(labels, r\PYZsq{}(?s)\PYZbs{}|citation:(.*?)\PYZbs{}|\PYZsq{}) AS citation,
    REGEXP\PYZus{}EXTRACT(labels, r\PYZsq{}\PYZbs{}|cite\PYZus{}key:(.*?)\PYZbs{}|\PYZsq{}) AS cite\PYZus{}key,
    REGEXP\PYZus{}EXTRACT(labels, r\PYZsq{}\PYZbs{}|metric\PYZus{}name:(.*?)\PYZbs{}|\PYZsq{}) AS metric\PYZus{}name,
    REGEXP\PYZus{}EXTRACT(labels, r\PYZsq{}(?s)\PYZbs{}|metric\PYZus{}summary:(.*?)\PYZbs{}|\PYZsq{}) AS metric\PYZus{}summary,
    REGEXP\PYZus{}EXTRACT(labels, r\PYZsq{}(?s)\PYZbs{}|metric\PYZus{}usage:(.*?)\PYZbs{}|\PYZsq{}) AS metric\PYZus{}usage,
    REGEXP\PYZus{}EXTRACT(labels, r\PYZsq{}(?s)\PYZbs{}|attack\PYZus{}graph\PYZus{}reduced:(.*?)\PYZbs{}|\PYZsq{}) AS attack\PYZus{}graph,
    REGEXP\PYZus{}EXTRACT(labels, r\PYZsq{}(?s)\PYZbs{}|transition\PYZus{}matrix:(.*?)\PYZbs{}|\PYZsq{}) AS transition\PYZus{}matrix,
    REGEXP\PYZus{}EXTRACT(labels, r\PYZsq{}(?s)\PYZbs{}|transition\PYZus{}matrix\PYZus{}raw:(.*?)\PYZbs{}|\PYZsq{}) AS transition\PYZus{}matrix\PYZus{}raw,
    
  FROM
    `cloud\PYZhy{}performance\PYZhy{}tool.test\PYZus{}notebook\PYZus{}bq.test1`
   WHERE
     test = \PYZsq{}mttf\PYZsq{}
     AND metric = \PYZsq{}mttf\PYZsq{} 
     and run\PYZus{}uri = \PYZsq{}abeb2a2f\PYZsq{}
    
    limit 1
     )
\end{Verbatim}
\end{tcolorbox}

    \begin{Verbatim}[commandchars=\\\{\}]
UsageError: Cell magic `\%\%bigquery` not found.
    \end{Verbatim}

    \begin{tcolorbox}[breakable, size=fbox, boxrule=1pt, pad at break*=1mm,colback=cellbackground, colframe=cellborder]
\prompt{In}{incolor}{ }{\boxspacing}
\begin{Verbatim}[commandchars=\\\{\}]

\end{Verbatim}
\end{tcolorbox}

    \begin{tcolorbox}[breakable, size=fbox, boxrule=1pt, pad at break*=1mm,colback=cellbackground, colframe=cellborder]
\prompt{In}{incolor}{ }{\boxspacing}
\begin{Verbatim}[commandchars=\\\{\}]

\end{Verbatim}
\end{tcolorbox}

    \begin{tcolorbox}[breakable, size=fbox, boxrule=1pt, pad at break*=1mm,colback=cellbackground, colframe=cellborder]
\prompt{In}{incolor}{ }{\boxspacing}
\begin{Verbatim}[commandchars=\\\{\}]

\end{Verbatim}
\end{tcolorbox}

    \begin{tcolorbox}[breakable, size=fbox, boxrule=1pt, pad at break*=1mm,colback=cellbackground, colframe=cellborder]
\prompt{In}{incolor}{ }{\boxspacing}
\begin{Verbatim}[commandchars=\\\{\}]

\end{Verbatim}
\end{tcolorbox}

    \begin{tcolorbox}[breakable, size=fbox, boxrule=1pt, pad at break*=1mm,colback=cellbackground, colframe=cellborder]
\prompt{In}{incolor}{ }{\boxspacing}
\begin{Verbatim}[commandchars=\\\{\}]

\end{Verbatim}
\end{tcolorbox}


    % Add a bibliography block to the postdoc
    
    
    
\end{document}
